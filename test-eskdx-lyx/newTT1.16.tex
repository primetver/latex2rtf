%% LyX 2.0.3 created this file.  For more info, see http://www.lyx.org/.
%% Do not edit unless you really know what you are doing.
\documentclass[12pt,a4paper,russian,vpadding=25mm, hpadding=10mm, pointsubsection, floatsection]{eskdtext}
\usepackage[T1]{fontenc}
\usepackage[utf8]{inputenc}
\usepackage{babel}
\usepackage{array}
\usepackage{longtable}
\usepackage{textcomp}
\usepackage{setspace}
\usepackage{nomencl}
% the following is useful when we have the old nomencl.sty package
\providecommand{\printnomenclature}{\printglossary}
\providecommand{\makenomenclature}{\makeglossary}
\makenomenclature
\onehalfspacing
\usepackage[unicode=true,pdfusetitle,
 bookmarks=true,bookmarksnumbered=false,bookmarksopen=false,
 breaklinks=false,pdfborder={0 0 0},backref=false,colorlinks=false]
 {hyperref}

\makeatletter

%%%%%%%%%%%%%%%%%%%%%%%%%%%%%% LyX specific LaTeX commands.
\pdfpageheight\paperheight
\pdfpagewidth\paperwidth

\DeclareRobustCommand{\cyrtext}{%
  \fontencoding{T2A}\selectfont\def\encodingdefault{T2A}}
\DeclareRobustCommand{\textcyr}[1]{\leavevmode{\cyrtext #1}}
\AtBeginDocument{\DeclareFontEncoding{T2A}{}{}}

%% Because html converters don't know tabularnewline
\providecommand{\tabularnewline}{\\}

%%%%%%%%%%%%%%%%%%%%%%%%%%%%%% Textclass specific LaTeX commands.
\newcommand{\strong}[1]{\textbf{#1}}

%%%%%%%%%%%%%%%%%%%%%%%%%%%%%% User specified LaTeX commands.
\usepackage{pgtz}
\input{eskdxrtf}

\makeatother

\begin{document}


\ESKDcompany{Центральный Банк Российской федерации \\(Банк России)}
\ESKDdocName{Техническое задание}
\ESKDsignature{\ESKDtheGroup.425790.036.М}
\ESKDtitleApprovingSheet{\ESKDtheSignature.ЛУ}
\newcommand{\sysabbr}{СИП СГА} %задать условное обозначение системы


\title{Система информационной поддержки деятельности Службы главного аудитора}

\maketitle

\section*{Аннотация}

Техническое задание (ТЗ)\nomenclature{ТЗ}{Техническое задание}
<<\ESKDtheTitle>> (\sysabbr) содержит согласованные между Заказчиком
и Исполнителем требования к архитектуре, техническим характеристикам,
функциональным возможностям, порядку разработки \sysabbr, к комплекту
документации на систему. Документ разработан и оформлен в соответствии
с требованиями ГОСТ~34.602--89 \cite{GOST34602:89} и ГОСТ~2.105--95.

\newpage{}

\tableofcontents{}\newpage{}


\section{Общие сведения }


\subsection{Полное наименование системы и ее условное обозначение }

Полное наименование: <<\ESKDtheTitle>>. 

Условное обозначение: \sysabbr. 


\subsection{Шифр темы }

Шифр темы: «\sysabbr». 


\subsection{Сведения о Заказчике и Исполнителе работы }

Заказчиком на проведение работы является Банк России, юридический
адрес: 107016, г.\,Москва, ул.\,Неглинная, д.\,12. 

Представитель Заказчика: Департамент информационных систем Банка России
(ДИС\nomenclature{ДИС}{Департамент информационных систем}). 

Исполнитель работы определяется в соответствии с требованиями Положения
Банка России от 25 марта 2008 года №\,317--П «О порядке выбора контрагентов
при заключении договоров Центральным банком Российской Федерации». 


\subsection{Перечень документов, на основании которых создается \sysabbr}

Основанием для проведения работы являются:
\begin{itemize}
\item Решение Департамента информационных систем Банка России о проведении
работ по созданию первой очереди \sysabbr{} в 2011--2012 гг. от 01.04.2011
№ ТР--1086;
\item документ «Архитектура и порядок создания системы информационной поддержки
деятельности службы главного аудитора Банка России», утвержденный
\_\_.10.2011. 
\end{itemize}

\subsection{Плановые сроки выполнения работы }

Состав, содержание и сроки выполнения работ определяются в разделе~\ref{sec:=000421=00043E=000441=000442=000430=000432}
настоящего ТЗ.


\subsection{Сведения об источниках финансирования }

Финансирование осуществляется за счет средств Заказчика. 


\subsection{Порядок уточнения и дополнения Технического задания}

ТЗ может уточняться и дополняться в процессе выполнения работы оформлением
дополнений к нему в соответствии с требованиями ГОСТ~34.602--89.
Согласование и утверждение дополнений к ТЗ должно проводиться в порядке,
установленном для ТЗ в ГОСТ~34.602--89.


\subsection{Порядок оформления и предъявления Заказчику результатов работ}

Описание порядка оформления и предъявления Заказчику результатов работ
по созданию \sysabbr{} приведено в разделе~\ref{sec:=00041A=00043E=00043D=000442=000440=00043E=00043B=00044C}.
Требования к составу и оформлению предъявляемой Заказчику документации
на \sysabbr{} приведены в разделе~\ref{sec:=000414=00043E=00043A=000443=00043C=000435=00043D=000442=000438=000440=00043E=000432=000430=00043D=000438=000435}.
Ввод в постоянную эксплуатацию \sysabbr{} осуществляется после проведения
предварительных испытаний, опытной эксплуатации и приемочных испытаний
системы.

\newpage{}


\section{Назначение и цели создания \sysabbr}


\subsection{Назначение \sysabbr }

\ESKDtheTitle{} (\sysabbr) предназначена для автоматизации деятельности
Службы главного аудитора Банка России (СГА\nomenclature{СГА}{Служба главного аудитора})
в части:
\begin{itemize}
\item планирования и подготовки проведения аудиторских проверок субъектами
внутреннего аудита Банка России;
\item информационной поддержки проведения аудиторских проверок;
\item мониторинга аудиторских проверок, проводимых субъектами внутреннего
аудита Банка России;
\item подготовки управленческих решений и реализации материалов аудиторских
проверок;
\item аналитической обработки информации, накапливаемой в процессе деятельности
СГА.
\end{itemize}

\subsection{Цели создания \sysabbr}

\pointОсновной целью создания \sysabbr{}\nomenclature{СИП}{Система информационной поддержки}
является повышение эффектив\-ности внутреннего аудита в Банке России
за счет использования в работе СГА средств автоматизации и связанных
с этим:
\begin{itemize}
\item снижением трудозатрат при выполнении СГА основных задач внутреннего
аудита;
\item возможностями централизованного накопления и аналитической обработки
данных;
\item повышением оперативности обработки данных;
\item обеспечением защиты данных от несанкционированного доступа в соответствии
со стандартом Банка России по обеспечению информационной безопасности.
\end{itemize}


\point\sysabbr{} разрабатывается в качестве основного средства автоматизации
деятельности СГА.

\newpage{}


\section{Характеристики объекта автоматизации}

\bigpointОбъектом автоматизации \sysabbr{} является деятельность
СГА, включающая следующие процессы:
\begin{itemize}
\item планирование аудиторских проверок;
\item подготовка аудиторских проверок;
\item проведение аудиторских проверок;
\item реализация материалов аудиторских проверок;
\item контроль устранения выявленных нарушений и недостатков;
\item мониторинг аудиторских проверок.
\end{itemize}




\bigpointПланирование аудиторских проверок включает:
\begin{itemize}
\item формирование годовых планов аудиторских проверок на риск-ориентированной
основе; 
\item планирование ресурсного обеспечения аудиторских проверок;
\item согласование и утверждение годовых планов аудиторских проверок;
\item контроль согласования и утверждения годовых планов аудиторских проверок.
\end{itemize}


\bigpointПри  планировании, определение состава аудиторских проверок
и планирование ресурсного обеспечения аудиторских проверок осуществляется
с учетом установленной Положением Банка России «О внутреннем аудите
в Центральном банке Российской Федерации» (далее - Положение) периодичности
проведения аудиторских проверок и риск-ориентированного подхода. .

\bigpointУчастниками процесса планирования аудиторских проверок
являются  подразделения СГА.

\bigpointПодготовка аудиторских проверок осуществляется в соответствии
с Положением.

\bigpointОрганизация проведения аудиторской проверки включает следующие
процедуры: 
\begin{itemize}
\item сбор и анализ информации о деятельности объекта аудита;
\item определение перечня проверяемых вопросов с учетом принципа риск-ориен\-ти\-ро\-ван\-но\-го
подхода;
\item назначение руководителя аудиторской проверки и формирование группы
проверяющих;
\item подготовка пакета документов, необходимого для проведения аудиторской
проверки, подписание документов уполномоченными лицами.
\end{itemize}




\bigpointПроведение аудиторской проверки включает следующие процедуры:
\begin{itemize}
\item сбор и анализ аудиторских доказательств;
\item оценка выполнения объектом аудита возложенных на него функций, соблюдения
законодательства Российской Федерации, нормативных и иных актов Банка
России, сохранности ценностей, активов, в том числе полученного обеспечения,
и эффективного использования ресурсов, достоверности бухгалтерской
(финансовой) и другой отчетности, состояния внутреннего контроля и
управления рисками;
\item формирование отчетов проверяющих;
\item формирование акта аудиторской проверки.
\end{itemize}


\bigpointРеализация материалов аудиторских проверок включает следующие
процедуры:
\begin{enumerate}
\item рассмотрение актов аудиторских проверок, планов мероприятий (далее
- материалы аудиторских проверок), включающее:

\begin{itemize}
\item направление в профильные департаменты Банка России выписки из материалов
аудиторской проверки по вопросам, относящимся к их компетенции (далее
- выписки);
\item получение от профильных департаментов Банка России предложений об
управленческом решении и мерах по устранению выявленных нарушений
и недостатков;
\item назначение аудиторской проверки при необходимости получения дополнительной
информации по конкретным вопросам;
\item анализ материалов аудиторских проверок и подготовка докладной записки
о результатах аудиторской проверки и рассмотрения ее материалов (далее
- докладная записка) и проекта управленческого решения;
\item формирование предложений о внесении изменений в акт аудиторской проверки
и подготовка проекта решения главного аудитора Банка России о внесении
изменений в акт аудиторской проверки;
\end{itemize}
\item Контроль за устранением объектами аудита выявленных нарушений и недостатков
принятых по результатам проведенных аудиторских проверок:

\begin{itemize}
\item получение и рассмотрение отчета о выполнении плана мероприятий;
\item доклад о результатах рассмотрения плана мероприятий и возможности
завершения работы с материалами аудиторской проверки;
\item направление в профильные департаменты Банка России выписки из отчета
о выполнении плана мероприятий по вопросам, относящимся к их компетенции,
для дальнейшего контроля за их выполнением.
\end{itemize}
\end{enumerate}


\bigpointМониторинг аудиторских проверок направлен  на формирование
и предоставление информационного обеспечения планирования аудиторских
проверок, реализации материалов аудиторских проверок и контроля устранения
выявленных нарушений и недостатков. 

\bigpointВ рамках мониторинга аудиторских проверок осуществляется:
\begin{itemize}
\item сбор, консолидация и обработка данных об аудиторских проверках и их
результатах;
\item консолидация планов аудиторских проверок;
\item консолидация планов мероприятий, направленных на устранение выявленных
нарушений и недостатков;
\item подготовка и предоставление отчетности в соответствии с существующими
регламентами, а также отчетности формируемой по требованию. 
\end{itemize}


\bigpointПри мониторинге аудиторских проверок осуществляется сбор
следующих видов сведений:
\begin{itemize}
\item о формировании и корректировке годовых планов аудиторских проверок;
\item об изменении статуса (состояния) годовых планов аудиторских проверок;
\item о ходе подготовки, проведения и реализации материалов аудиторских
проверок;
\item о документах, выпускаемых при подготовке аудиторской проверки;
\item отчетов об аудиторских проверках, содержащих данные о выявленных нарушениях
и недостатках;
\item актов аудиторских проверок;
\item планов мероприятий по устранению недостатков;
\item о заключениях профильных департаментов Банка России по выявленным
нарушениям и недостаткам;
\item о принятых управленческих решениях;
\item о выполнении планов мероприятий по устранению выявленных нарушений
и недостатков.
\end{itemize}
\newpage{}


\section{Требования к \sysabbr}


\subsection{Требования к \sysabbr{} в целом}


\subsubsection{\label{sub:=000422=000440=000435=000431=00043E=000432=000430=00043D=000438=00044F-=00043A-=000441=000442=000440=000443=00043A=000442=000443=000440=000435}Требования
к структуре и функционированию \sysabbr}

\smallpointВ составе \sysabbr{} должны быть предусмотрены:
\begin{itemize}
\item функциональные подсистемы (ФП\nomenclature{ФП}{Функциональная подсистема});
\item обеспечивающие подсистемы (ОП\nomenclature{ОП}{Обеспечивающая подсистема});
\item подсистема информационной безопасности (ПИБ\nomenclature{ПИБ}{Подсистема информационной безопасности});
\item комплекс программно-технических средств (КПТС\nomenclature{КПТС}{Комплекс программно-технических средств}).
\end{itemize}


\smallpointВ состав ФП \sysabbr{} должны входить:
\begin{itemize}
\item ФП планирования аудиторских проверок;
\item ФП поддержки подготовки аудиторских проверок;
\item ФП поддержки проведения аудиторских проверок;
\item ФП поддержки реализации материалов аудиторской проверки и контроля
за устранением нарушений и недостатков;
\item ФП мониторинга аудиторских проверок;
\item ФП внутреннего документооборота СГА;
\item ФП информационных ресурсов СГА.
\end{itemize}


\smallpoint\label{=000424=00041F-=00043F=00043B=000430=00043D=000438=000440=00043E=000432=000430=00043D=000438=00044F-=000430=000443=000434=000438=000442=00043E=000440=000441=00043A=000438=000445}ФП
планирования аудиторских проверок должна включать задачи:
\begin{itemize}
\item Риск-ориентированное планирование аудиторских проверок;
\item Ведение планов аудиторских проверок;
\item Контроль и согласование планов аудиторских проверок.
\end{itemize}


\smallsubpointВ составе первой очереди \sysabbr{} должны быть реализованы
функции задач ФП планирования аудиторских проверок, перечисленных
в п. \ref{sub:=000422=000440=000435=000431=00043E=000432=000430=00043D=000438=00044F-=00043A-=000441=000442=000440=000443=00043A=000442=000443=000440=000435}.\ref{=000424=00041F-=00043F=00043B=000430=00043D=000438=000440=00043E=000432=000430=00043D=000438=00044F-=000430=000443=000434=000438=000442=00043E=000440=000441=00043A=000438=000445}
настоящего технического задания. 



\smallsubpointТребования к характеристикам и функциям ФП планирования
аудиторских проверок первой очереди \sysabbr{} приведены в п. \ref{sub:=000422=000440=000435=000431=00043E=000432=000430=00043D=000438=00044F-=00043A-=000424=00041F1}
настоящего технического задания.

\smallpoint\label{=000424=00041F-=00043F=00043E=000434=000434=000435=000440=000436=00043A=000438-=00043F=00043E=000434=000433=00043E=000442=00043E=000432=00043A=000438-1}ФП
поддержки подготовки аудиторских проверок должна включать задачи:
\begin{enumerate}
\item Формирование пакета документов по аудиторской проверке; 
\item Формирование состава групп проверяющих:
\item Подготовка справки по объекту аудита;
\item Формирование справочной информации о требованиях к деятельности, осуществляемой
объектом аудита;
\item Формирование справки о внутреннем контроле на объекте  аудита;
\item Формирование справки по источникам исходной информации.
\end{enumerate}
\smallsubpointЗадача Формирование пакета документов по аудиторской
проверке должна обеспечивать автоматизацию:
\begin{itemize}
\item формирования документов по аудиторской проверке на основе шаблонов;
\item включения в формируемые документы информации о виде аудиторской проверки,
объекте  аудита, руководителе группы проверяющих; 
\item выгрузки документов по аудиторской проверке на машиночитаемый носитель;
\item печати документов по аудиторской проверке.
\end{itemize}
\smallsubpointЗадача Формирование состава групп проверяющих должна
обеспечивать автоматизацию:
\begin{itemize}
\item формирования состава групп проверяющих на основе информации о виде
аудиторской проверки, требуемых сроков проведения аудиторской проверки,
составе проверяемых вопросов, занятости сотрудников;
\item согласования состава группы проверяющих; 
\item включения согласованного состава группы проверяющих в справку о подготовке
аудиторской проверки.
\end{itemize}
\smallsubpointЗадача Подготовка справки по объекту аудита должна
обеспечивать автоматизацию:
\begin{itemize}
\item формирования справки по объекту аудита;
\item включения сформированных сведений в справку о подготовке аудиторских
проверок.
\end{itemize}
\smallsubpointФормируемая справка по объекту аудита должна включать
сведения:
\begin{itemize}
\item о проводившихся ранее на объекте аудита проверках и их результатах;
\item об особенностях объекта (объектов) аудита и рисках, связанных с деятельностью,
осуществляемой на объекте аудита; 
\item о реализовавшихся рисковых событиях на объекте (объектах) аудита;
\item о мероприятиях по устранению нарушений и недостатков, выявленных на
объекте (объектах) аудита.
\end{itemize}
\smallsubpointЗадача Формирование справочной информации о требованиях
к деятельности, осуществляемой объектом аудита, должна обеспечивать
автоматизацию:
\begin{itemize}
\item формирования сведений о составе документов нормативно-методической
базы, определяющих требования к деятельности, осуществляемой на объекте
(объектах) аудита;
\item подготовки выписок требований к деятельности объекта (объектов внутреннего
аудита) из документов НМБ Банка России;
\item выгрузки подготовленной информации о требованиях к деятельности объекта
аудита на машиночитаемый носитель.
\end{itemize}
\smallsubpointЗадача Формирование справки о внутреннем контроле на
объекте  аудита должна обеспечивать автоматизацию формирования сведений:
\begin{itemize}
\item об организации внутреннего контроля на объекте аудита;
\item о состоянии внутреннего контроля на объекте (объектах) аудита;
\item о процедурах внутреннего контроля, выполняемых в отношении деятельности
объекта аудита.
\end{itemize}
\smallsubpointЗадача Формирование справки по источникам исходной
информации должна обеспечивать автоматизацию формирования: 
\begin{itemize}
\item перечня документов, информационных систем, которые могут быть использованы
в качестве источников для получения аудиторских доказательств по проверяемым
направлениям деятельности объекта (объектов) аудита;
\item сведений об оценке актуальности информации источника.
\end{itemize}






\smallsubpointВ составе первой очереди \sysabbr{} должны быть реализованы
функции следующих задач ФП поддержки подготовки аудиторских проверок:
\begin{itemize}
\item Формирование пакета документов по аудиторской проверке;
\item Формирование состава групп проверяющих;
\item Подготовка справки по объекту внутреннего аудита.
\end{itemize}


\smallsubpointТребования к характеристикам и функциям ФП поддержки
подготовки аудиторских проверок первой очереди \sysabbr{} приведены
в п. \ref{sub:=000422=000440=000435=000431=00043E=000432=000430=00043D=000438=00044F-=00043A-=000424=00041F2}
настоящего технического задания.



\smallpointФП поддержки проведения аудиторских проверок создается
в рамках второй и третьей очередей \sysabbr{}.

\smallpoint\label{=000424=00041F-=00043F=00043E=000434=000434=000435=000440=000436=00043A=000438-=00043F=00043E=000434=000433=00043E=000442=00043E=000432=00043A=000438}ФП
поддержки реализации материалов аудиторской проверки и контроля за
устранением нарушений и недостатков:
\begin{enumerate}
\item Ввод актов аудиторских проверок;
\item Поддержка рассмотрения материалов аудиторской проверки и подготовки
проектов управленческих решений;
\item Импорт планов мероприятий по устранению нарушений и недостатков, выявленных
аудиторской проверкой (далее – план мероприятий);
\item Импорт отчетов о выполнении плана мероприятий;
\item Контроль выполнения планов мероприятий.
\end{enumerate}
\smallsubpointЗадача Ввод актов аудиторских проверок, должна обеспечивать
автоматизацию:
\begin{itemize}
\item ввода и сохранения в структурированном виде актов аудиторских проверок;
\item ввода и сохранения изменений актов аудиторских проверок;
\item ведение версий актов аудиторских проверок.
\end{itemize}
\smallsubpointЗадача Поддержка рассмотрения материалов аудиторской
проверки и подготовки проектов управленческих решений, должна обеспечивать
автоматизацию:
\begin{itemize}
\item подготовки выписок из материалов аудиторских проверок;
\item получения, сохранения и использования предложений об управленческом
решении и мерах по устранению выявленных нарушений и недостатков;
\item подготовки пояснительных записок по результатам рассмотрения материалов
аудиторских проверок;
\item подготовки проектов управленческих решений.
\end{itemize}
\smallsubpointЗадача Импорт планов мероприятий должна обеспечивать
автоматизацию:
\begin{itemize}
\item ввода и сохранения планов мероприятий;
\item ввода информации о принятых по результатам аудиторских проверок управленческих
решениях;
\item ввода корректировок планов мероприятий;
\item ведения версий планов мероприятий.
\end{itemize}
\smallsubpointЗадача Импорт отчетов о выполнении плана мероприятий
должна обеспечивать автоматизацию:
\begin{itemize}
\item ввода и сохранения в структурированном виде отчетов о выполнении планов
мероприятий и выполнении управленческих решений;
\item ведения версий отчетов о выполнении планов мероприятий и выполнении
управленческих решений.
\end{itemize}
\smallsubpointЗадача Контроль выполнения планов мероприятий должна
обеспечивать автоматизацию:
\begin{itemize}
\item контроля выполнения планов мероприятий на основе введенных отчетов
о выполнении мероприятий;
\item формирования требований к включению в план следующего года аудиторских
проверок объектов аудита, если по результатам проведенной аудиторской
проверки принято управленческое решение в форме приказа по основной
деятельности или представления;
\item учета результатов аудиторских проверок по контролю устранения нарушений
и недостатков.
\end{itemize}






\smallsubpointВ составе первой очереди \sysabbr{} должны быть реализованы
функции следующих задач ФП поддержки реализации материалов аудиторской
проверки и контроля за устранением нарушений и недостатков:
\begin{enumerate}
\item Ввод актов аудиторских проверок;
\item Импорт планов мероприятий;
\item Импорт отчетов о выполнении плана мероприятий;
\item Контроль выполнения планов мероприятий.
\end{enumerate}


\smallsubpointТребования к характеристикам и функциям ФП поддержки
реализации материалов аудиторской проверки и контроля за устранением
нарушений и недостатков, реализуемой на первой очереди \sysabbr{},
приведены в п. \ref{sub:=000422=000440=000435=000431=00043E=000432=000430=00043D=000438=00044F-=00043A-=000424=00041F3}
настоящего технического задания.



\smallpoint\label{=000424=00041F-=00043C=00043E=00043D=000438=000442=00043E=000440=000438=00043D=000433=000430-=000430=000443=000434=000438=000442=00043E=000440=000441=00043A=000438=000445-1}ФП
мониторинга аудиторских проверок должна включать задачи:
\begin{enumerate}
\item Ввод и передача в центральное звено отчетов об аудиторских проверках;
\item Анализ структуры нарушений и недостатков;
\item Формирование и анализ сводного плана аудиторских проверок;
\item Анализ выполнения планов аудиторских проверок;
\item Анализ динамики выявления нарушений;
\item Анализ эффективности мероприятий по устранению выявленных нарушений;
\item Поддержка выявления массовых и системных нарушений;
\item Мониторинг состояния аудиторских проверок.
\end{enumerate}


\smallsubpointЗадачи ФП мониторинга аудиторских проверок должны обеспечивать
автоматизацию формирования отчетности в соответствии с существующими
регламентами и формирование настраиваемых отчетов.



\smallsubpointВ составе первой очереди \sysabbr{} должны быть реализованы
следующие задачи ФП мониторинга аудиторских проверок:
\begin{enumerate}
\item Ввод и передача в центральное звено отчетов об аудиторских проверках;
\item Анализ структуры нарушений и недостатков;
\item Формирование и анализ сводного плана аудиторских проверок;
\item Анализ выполнения планов аудиторских проверок;
\item Анализ динамики выявления нарушений;
\item Мониторинг состояния аудиторских проверок.
\end{enumerate}


\smallsubpointТребования к ФП мониторинга аудиторских проверок, реализуемые
в составе первой очереди \sysabbr{} приведены в п. \ref{sub:=000422=000440=000435=000431=00043E=000432=000430=00043D=000438=00044F-=00043A-=000424=00041F4}
настоящего технического задания.



\smallpointФП внутреннего документооборота СГА должна быть создана
в составе первой очереди \sysabbr{}. 

\smallsubpointТребования к ФП внутреннего документооборота СГА, реализуемые
в составе первой очереди \sysabbr{} приведены п. \ref{sub:=000422=000440=000435=000431=00043E=000432=000430=00043D=000438=00044F-=00043A-=000424=00041F5}
настоящего технического задания.

\smallpoint\label{=000424=00041F-=000440=000435=000435=000441=000442=000440-=000438=00043D=000444=00043E=000440=00043C=000430=000446=000438=00043E=00043D=00043D=00044B=000445}ФП
информационных ресурсов СГА должна включать задачи:
\begin{enumerate}
\item Ведение данных о квалификации и занятости сотрудников СГА;
\item Ведение реестра аудиторских проверок;
\item Ведение архива материалов аудиторских проверок;
\item Ведение шаблонов документов;
\item Ведение справочников;
\item Ведение каталога объектов аудита;
\item Накопление информации по объектам аудита;
\item Ведение базы нормативных актов;
\item Ведение методического обеспечения аудиторских проверок.
\end{enumerate}


\smallsubpointЗадача Ведение данных о квалификации и занятости сотрудников
СГА должна обеспечивать автоматизацию ввода и актуализации:
\begin{itemize}
\item сведений о квалификации и занятости сотрудников СГА;
\item оценок деятельности сотрудников СГА.
\end{itemize}
\smallsubpointЗадача Ведение реестра аудиторских проверок должна
обеспечивать автоматизацию ввода и актуализации:
\begin{itemize}
\item регистрационной информации по аудиторским проверкам;
\item сведений о ходе подготовки, проведения и реализации материалов аудиторских
проверок;
\item информации об устранении нарушений и недостатков, выявленных аудиторскими
проверками.
\end{itemize}
\smallsubpointЗадача Ведение архива материалов аудиторских проверок
должна обеспечивать ввод и сохранение документов, формируемых при
подготовке, проведении, реализации материалов аудиторских проверок
и контроле устранения нарушений и недостатков. 

\smallsubpointЗадача Ведение шаблонов документов должна обеспечивать
автоматизацию ввода, редактирования и хранения истории изменений шаблонов:
\begin{itemize}
\item документов формируемых при подготовке аудиторских проверок;
\item отчетов проверяющих и актов аудиторских проверок;
\item документов формируемых при реализации материалов аудиторских проверок.
\end{itemize}
\smallsubpointЗадача Ведение справочников должна обеспечивать автоматизацию
ввода, корректировки и хранения истории изменений справочников СИП
СГА. 

\smallsubpointЗадача Ведение каталога объектов аудита должна обеспечивать
автоматизацию:
\begin{itemize}
\item ведения перечня объектов аудита;
\item ведения описаний объектов аудита.
\end{itemize}
\smallsubpointЗадача Накопление информации по объектам аудита должна
обеспечивать автоматизацию:
\begin{itemize}
\item ведения информации о рисках, связанных с деятельностью, осуществляемой
на объектах аудита;
\item ведение информации о внутреннем контроле на объектах аудита;
\item ведение информации о документах, информационных системах и других
источниках контрольной информации по объектам аудита. 
\end{itemize}
\smallsubpointЗадача Ведение базы нормативных актов должна обеспечивать
автоматизацию импорта из внешних источников и отслеживание изменений
нормативных документов, используемых в деятельности СГА.

\smallsubpointЗадача Ведение методического обеспечения аудиторских
проверок должна обеспечивать автоматизацию ввода, корректировки и
хранения истории изменений документов, составляющих методическое обеспечение
организации и проведения аудиторских проверок.



\smallsubpointВ составе первой очереди \sysabbr{} должны быть реализованы
следующие задачи ФП информационных ресурсов СГА:
\begin{enumerate}
\item Ведение данных о квалификации и занятости сотрудников СГА;
\item Ведение реестра аудиторских проверок;
\item Ведение архива материалов аудиторских проверок;
\item Ведение шаблонов документов;
\item Ведение справочников;
\item Ведение каталога объектов аудита.
\end{enumerate}


\smallsubpointТребования к ФП информационных ресурсов первой очереди
\sysabbr{} приведены в п. \ref{sub:=000422=000440=000435=000431=00043E=000432=000430=00043D=000438=00044F-=00043A-=000424=00041F6}
настоящего технического задания.



\smallpointВ состав ОП \sysabbr{} должны входить:
\begin{itemize}
\item ОП информационного хранилища;
\item ОП рабочего документооборота;
\item ОП внутрисистемного взаимодействия;
\item ОП администрирования.
\end{itemize}


\smallpoint\sysabbr{} должна быть реализована как многопользовательская
клиент-серверная регионально-распределенная система, включающая следующие
звенья:
\begin{itemize}
\item Центральное звено, предназначенное для использования в ДВА (далее
--- ЦЗ);
\item Региональное звено, предназначенное для использования в других подразделениях
СГА (далее --- РЗ).
\end{itemize}
\smallpointВ состав КПТС \sysabbr{} должны входить:
\begin{itemize}
\item серверы \sysabbr{};
\item автоматизированные рабочие места (далее --- АРМ) пользователей СИП
СГА.
\end{itemize}
\smallpointВ состав серверов \sysabbr{} должны быть включены:
\begin{itemize}
\item сервер информационного хранилища \sysabbr{};
\item сервер доступа центрального звена \sysabbr{}.
\end{itemize}


\smallpointВ состав \sysabbr{} должны входить следующие виды АРМ:
\begin{itemize}
\item стационарные;
\item мобильные.
\end{itemize}


\smallpointСтационарные АРМ должны представлять собой персональные
компьютеры, оснащенные СЗИ\nomenclature{СЗИ}{Средство защиты информации}
от НСД\nomenclature{НСД}{Несанкционированный доступ},
с установленным программным обеспечением (далее — ПО\nomenclature{ПО}{Программное обеспечение})
\sysabbr{}. Мобильные рабочие места пользователей \sysabbr{} должны
представлять собой ноутбуки с установленным ПО АРМ \sysabbr{} и СЗИ
от НСД. 
\begin{description}
\item [{Примечание}] — Мобильные  АРМ предназначены для обеспечения автоматизации
деятельности проверяющих. 
\end{description}
\smallpointАРМ должны быть реализованы по одной из следующих технологий:
\begin{itemize}
\item технология «толстого клиента», т.е. с установкой специализированного
прикладного ПО на АРМ пользователя;
\item технология «тонкого клиента», т.е. с доступом к функциям \sysabbr{}
с использованием веб-браузера и без установки специализированного
прикладного ПО на АРМ пользователя. 
\end{itemize}


\smallpointАРМ  РЗ должны быть реализованы как автономные АРМ, с
использованием технологий пакетного обмена с серверными компонентами
\sysabbr{}. Должно обеспечиваться функционирование автономных АРМ
\sysabbr{} независимо от наличия сетевого подключения к средствам
центрального звена \sysabbr{}. 

\smallpointАРМ  предназначенные для использования только в ДВА,
должны разрабатываться по технологии «тонкого клиента». 

\smallpointВнутрисистемные пакетные взаимодействия \sysabbr{} должны
осуществляться через транспортные системы прикладного уровня (ТСПУ\nomenclature{ТСПУ}{Транспортная система прикладного уровня})
Банка России:
\begin{itemize}
\item транспортная среда доставки сообщений Единой транспортной системы
для обеспечения обмена электронными сообщениями в Банке России (СДС\nomenclature{СДС}{Транспортная среда доставки сообщений Единой транспортной системы для обеспечения обмена электронными сообщениями в Банке России});
\item система электронной почты Банка России. 
\end{itemize}


\smallpointПри создании \sysabbr{} должны быть разработаны решения
по интеграции в \sysabbr{} реализованных в рамках АС\nomenclature{АС}{Автоматизированная система}
МРП\nomenclature{МРП}{Мониторинг ревизий и проверок}
задач и функций, в части:
\begin{itemize}
\item обеспечения управления пользователями \sysabbr{};
\item управления правами доступа пользователей; 
\item ведения справочников; 
\item ведения планов аудиторских проверок ТУ;
\item ведения информации о сотрудниках СГА;
\item накопления информации мониторинга аудиторских проверок;
\item контроля подготовки аудиторских проверок; 
\item формирования документов, обеспечивающих проведение аудиторских проверок;
\item ведения шаблонов документов; 
\item импорта отчетности по форме 0409039; 
\item импорта планов аудиторских проверок.
\end{itemize}


\smallpointВ составе \sysabbr{} должны быть предусмотрены производственный
и тестовый участки. Промышленная эксплуатация \sysabbr{} должна осуществляться
в рамках производственного участка. Тестовый участок \sysabbr{} должен
быть изолирован (на уровне сетевого доступа в соответствии с моделью
OSI) от производственного участка. 

\smallpointТестовый участок \sysabbr{} должен состоять из стенда
ЦЗ и стенда РЗ. Состав прикладного ПО и системного ПО, развернутых
на тестовом участке \sysabbr{}, должен соответствовать составу 
ПО производственного участка \sysabbr{}. 

\smallpointТестовый участок \sysabbr{} должен обеспечивать возможность
проверки всех функций (задач) \sysabbr{}.

\smallpointВ состав \sysabbr{} должны быть включены средства мониторинга
событий ИБ компонентов системно-технического уровня, для обеспечения
контроля соответствующих событий со стороны АИБ \sysabbr{} в ДВА.

\smallpointСоздание комплекса технических средств (КТС) \sysabbr{},
разработка обеспечивающих подсистем \sysabbr{} и интеграция в \sysabbr{}
задач и функций, разработанных в рамках АС МРП должны быть осуществлены
в рамках создания первой очереди \sysabbr{}.


\subsubsection{Требования к совместимости}

\smallpointКлиентская часть программного обеспечения \sysabbr{}
должна функционировать в среде операционной системы (ОС\nomenclature{ОС}{Операционная система})
Microsoft Windows XP SP3, Microsoft Windows 7 и должна быть совместима
с Microsoft Internet Explorer версии 6.0, 7.0 и 8.0, а также техническими
и программными средствами защиты информационной безопасности, используемыми
в Банке России (СЗИ от НСД Аккорд и SecretNet , СКЗИ\nomenclature{СКЗИ}{Система криптографической защиты  информации}
«Форт», Kaspersky Anti-Virus, ПО «Паспорт ПО»).

\smallpointСерверные программные компоненты центрального звена \sysabbr{}
должны функционировать в среде ОС Microsoft Windows 2003 Server и
должны быть совместимы с Kaspersky Anti--Virus, серверным ПО СКЗИ
«Форт», ПО «Паспорт ПО».

\smallpointВ \sysabbr{} должны быть реализованы решения по обеспечению
взаимодействия между компонентами центрального и регионального звеньев
как с использованием СДС, так и с использованием системы электронной
почты Банка России. Основным каналом связи должна являться СДС, резервным
— система электронной почты Банка России. Операции по перенастройке
\sysabbr{} на работу с основным или резервным каналом связи не должны
занимать более 60 минут. При взаимодействии \sysabbr{} с СДС и системой
электронной почты Банка России должны использоваться протоколы SMTP\nomenclature{SMTP}{Simple Mail Transfer Protocol — простой протокол электронной почты}
и POP3\nomenclature{POP3}{Post Office Protocol Version 3 — протокол почтового отделения, версия 3}. 


\subsubsection{Требования к численности и квалификации персонала \sysabbr{} и режиму
его работы}

\smallpointПерсонал \sysabbr{} должен включать в себя:
\begin{itemize}
\item пользователей — специалистов СГА, эксплуатирующих \sysabbr{} в части
формирования, передачи, обработки и использования информации, а также
администратора системы (сотрудника ДВА, отвечающего за администрирование
\sysabbr{}) и администратора информационной безопасности (сотрудника
ДВА, отвечающего за администрирование средств и механизмов защиты
\sysabbr{});
\item обслуживающий персонал — специалистов подразделений информатизации
Банка России, обеспечивающих системно-техническое сопровождение \sysabbr{}.
\end{itemize}


\smallpointК квалификации пользователей (за исключением администратора
системы и администратора информационной безопасности) \sysabbr{}
должны предъявляться следующие требования:
\begin{itemize}
\item знания, достаточные для подготовки отчетов о проводимых (проведенных)
аудиторских проверках, и интерпретации отчетов об аудиторских проверках
и сводных отчетов;
\item знания, достаточные для соблюдения инструкций по обеспечению информационной
безопасности;
\item владение базовыми навыками работы с компьютером, использования ОС
семейства Microsoft Windows и работы с веб-браузером Microsoft Internet
Explorer; 
\item знание порядка работы с \sysabbr{} в объеме сведений, приведенных
в документах «Руководство пользователя» и «Общее описание системы». 
\end{itemize}


\smallpointКвалификация администратора системы должна быть достаточной
для эксплуатации и администрирования ПО \sysabbr{}, соблюдения инструкций
по обеспечению ИБ, а также знание порядка работы с \sysabbr{} в объеме
сведений, приведенных в документах «Руководство администратора» и
«Общее описание системы».

\smallpointК квалификации АИБ должны предъявляться следующие требования:
\begin{itemize}
\item знание схемы обработки данных в \sysabbr{};
\item для АИБ центрального звена --- понимание принципов работы СКЗИ «ФОРТ»
и навыки работы с СКЗИ «ФОРТ»;
\item владение навыками определения источников НСД и принятию соответствующих
контрмер, защищающих \sysabbr{} от выявленных источников НСД;
\item знания средств защиты информации от НСД;
\item навыки работы с СЗИ от НСД;
\item знание нормативных документов по обеспечению ИБ в Банке России;
\item навыки по определению источника сбоя функционирования программно--ап\-па\-рат\-ных
средств комплекса средств ПИБ \sysabbr{} и устранению сбоев;
\item знание программного обеспечения, осуществляющего антивирусную защиту;
\item знание принципов антивирусной защиты.
\end{itemize}


\smallpointКвалификация обслуживающего персонала \sysabbr{} должна
быть достаточной для системно--технического сопровождения \sysabbr{}.

\smallpointСостав, численность и режим работы обслуживающего персонала
\sysabbr{} и степень предметной подготовки должны определяться на
этапе технического проектирования \sysabbr{}.


\subsubsection{Требования к надежности}

\smallpointПроектными решениями должна обеспечиваться круглосуточная
доступность функций \sysabbr{}, за исключением случаев остановки
и перезапуска компонентов \sysabbr{} для выполнения сервисных работ.
Работоспособность \sysabbr{} должна автоматически восстанавливаться
после завершения перезапуска. 

\smallpointСбои в работе ПО и КТС \sysabbr{}, телекоммуникационной
инфраструктуры или сетей электроснабжения не должны приводить к внесению
искажений в данные, хранимые в \sysabbr{}. 

\smallpointСохранность и доступность хранимых в \sysabbr{} данных
должна обеспечиваться комплексом технических и организационных решений,
включая применение отказоустойчивого аппаратного обеспечения и организацию
резервного копирования данных, которые должны быть описаны в документе
«Пояснительная записка».

\smallpointПроектные решения должны предусматривать устранение отказов
в работе ПО и КТС центрального звена \sysabbr{} в течение 24 часов
с момента обнаружения отказа.

В \sysabbr{} должны быть выработаны и реализованы решения по повышению
устойчивости КПТС центрального звена \sysabbr{} к возможным отказам
аппаратного и системного программного обеспечения. Механизмом обеспечения
отказоустойчивости КПТС центрального звена \sysabbr{} должно являться
резервирование сервера информационного хранилища \sysabbr{} и сервера
доступа центрального звена \sysabbr{}.

\smallpointДолжна быть обеспечена автоматическая репликация хранимых
в \sysabbr{} данных с основного сервера информационного хранилища
\sysabbr{} на резервный сервер в оперативном режиме, по мере внесения
изменений в данные, хранимые в \sysabbr{} при ее функционировании.

\textbf{Примечание} --- Реализация функций репликации установленного
на серверах программного обеспечения и его настроек между основными
и резервными серверами \sysabbr{} настоящим ТЗ не предусматривается. 

\smallpointОперативная репликация данных с основного сервера информационного
хранилища \sysabbr{} на резервный должна осуществляться через серверный
сегмент локальной вычислительной сети (ЛВС) Центра информационных
технологий Банка России (ЦИТ). В случае размещения резервных серверов
\sysabbr{} на площадке, не имеющей непосредственного подключения
к серверному сегменту ЛВС ЦИТ, должна быть обеспечена криптографическая
защита данных, передаваемых между основным и резервным серверами
информационного хранилища \sysabbr{}. 

\smallpointОписание решений по обеспечению отказоустойчивости, включающее
в себя сведения о составе аппаратного и программного обеспечения \sysabbr{},
используемого для повышения отказоустойчивости КПТС центрального звена
\sysabbr{}  должно быть приведено в технорабочей документации \sysabbr{}. 

\smallpointВремя выполнения технологических операций по переносу
функционирования \sysabbr{} с основного комплекта серверов \sysabbr{}
на резервные серверы должно составлять не более 30 минут.

\smallpointДолжны быть разработаны и описаны в документации \sysabbr{}
следующие процедуры по переносу функционирования \sysabbr{} с основного
комплекта серверов на резервный:
\begin{enumerate}
\item процедура восстановления при аварийной ситуации (аварийное переключение)
– на случай неработоспособности основного сервера доступа и/или основного
сервера информационного хранилища; 
\item процедура переключения для выполнения регламентных работ (плановое
переключение) – на случай необходимости плановой остановки основного
сервера доступа и/или основного сервера информационного хранилища.
\end{enumerate}
\smallpointПроцедура переключения для выполнения регламентных работ
должна обеспечивать полную синхронизацию данных \sysabbr{} на резервном
сервере информационного хранилища с данными на основном сервере до
выполнения переключения. \smallpointВ случае применения процедуры
восстановления при аварийной ситуации допускается потеря обновлений
данных \sysabbr{} за период, соответствующий периоду запаздывания
обновлений данных \sysabbr{} на резервном сервере информационного
хранилища.

\textbf{Примечание} --- Требования к дополнительному оборудованию
и системному программному обеспечению, необходимым для реализации
резервирования серверов \sysabbr{}, приведены в подразделе \ref{sub:=000422=000440=000435=000431=00043E=000432=000430=00043D=000438=00044F-=00043A-=00041E=000431=000435=000441=00043F}.


\subsubsection{Требования к эксплуатации и техническому обслуживанию}

\smallpointСистемно-техническое сопровождение \sysabbr{} должно
проводиться обслуживающим персоналом \sysabbr{}.

\smallpointВыполнение модификаций программного обеспечения и структуры
базы данных (БД) должны осуществляться техническими специалистами
из состава обслуживающего персонала \sysabbr{} (возможно, с участием
специалистов подрядных организаций).



\smallpointПроцедура обновления программного обеспечения \sysabbr{},
определяющая порядок обновления прикладного ПО, справочников \sysabbr{}
и структуры БД, проведения тестирования изменений на тестовом участке
\sysabbr{} и последующего внесения изменений на производственный
участок \sysabbr{}, должна быть описана в следующих документах:
\begin{itemize}
\item «\sysabbr{}. Пояснительная записка к техническому проекту»;
\item «\sysabbr{}. Руководство администратора системы».
\end{itemize}
Процедуры планового и аварийного переключения с основного комплекта
серверов производственного участка \sysabbr{} на резервные сервера
должны выполняться специалистами ЦИТ. 

\smallpointДолжно быть обеспечено взаимодействие \sysabbr{} с пунктом
управления интегрированной системой управления телекоммуникационными
и информационными ресурсами ЦИТ (ПУ ИСУ ТИР ЦИТ) в части мониторинга
технических средств и системного ПО \sysabbr{} в порядке, установленном
в ЦИТ. 

\smallpointСведения об интеграции АС МРП с ПУ ИСУ ТИР ЦИТ должны
быть приведены в документе «Описание комплекса технических средств».




\subsubsection{\label{sub:=000422=000440=000435=000431=00043E=000432=000430=00043D=000438=00044F-=00043A-=000437=000430=000449=000438=000442=000435}Требования
к защите информации от несанкционированного доступа и обеспечению
информационной безопасности}

\strong{\smallpointОбщие требования по защите информации \sysabbr{}}

\smallsubpoint\sysabbr{} предназначена для обработки, хранения и
(или) передачи следующих категорий защищаемой информации:
\begin{itemize}
\item информация ограниченного доступа;
\item персональные данные сотрудников Банка России.
\end{itemize}


\sysabbr{} создается как АС четвертого типа в соответствии с классификацией,
введенной в Положении Банка России от 11.01.2008 № 316--П «Об обеспечении
информационной безопасности автоматизированных систем Банка России,
предназначенных для обработки, хранения и (или) передачи информации
ограниченного распространения» (далее по тексту — Положение 316--П).

\sysabbr{} является специальной ИСПДн, предназначенной для обработки
специальных категорий персональных данных (Далее ИСПДн-С) в соответствии
требованиями Временного порядка обеспечения безопасности персональных
данных, обрабатываемых в информационных системах персональных данных
Банка России (Приложение 2 к Приказу Банка России от 09.10.2009 №
ОД--650).

\smallsubpointВ \sysabbr{} не должна обрабатываться платежная информация. 

\smallsubpointЦелью обработки персональной информации в \sysabbr{}
в соответствии с Временным порядком обработки персональных данных
в информационных системах персональных данных Банка России (Приложение
1 к Приказу Банка России от 09.10.2009 № ОД--650) является осуществление
возложенных на Банк России законодательством Российской Федерации
функций, в том числе надзорных и контрольных, отраженных в Уведомлении
об обработке ПДн, в соответствии с федеральными законами, в частности:
«О Центральном банке Российской Федерации (Банке России)», «О банках
и банковской деятельности», «О кредитных историях», «О валютном регулировании
и валютном контроле», «О несостоятельности (банкротстве) кредитных
организаций», «О персональных данных», а так же принятыми в их исполнение
нормативными актами Банка России.

\smallsubpoint\sysabbr{} должна включать в себя подсистему информационной
безопасности (ПИБ), представляющую собой комплекс организационных,
технологических, технических и программных мер, средств и механизмов
защиты информации от несанкционированного доступа (далее -- НСД),
а также проектную и эксплуатационную документацию в соответствии с
нормативными документами Банка России.

\smallsubpointТребования по обеспечению информационной безопасности
должны реализовываться специализированной организацией, имеющей лицензию
Федеральной службы по техническому и экспортному контролю на осуществление
деятельности по технической защите конфиденциальной информации.

\smallsubpoint\sysabbr{} должна быть обеспечена проектной и эксплуатационной
документацией, отвечающей по видам, комплектности, обозначению и содержанию
требованиям ГОСТ 34.201–89, ГОСТ 34.602–89 и Руководящего документа
по стандартизации РД 50-34.698–90.
\begin{description}
\item [{Примечание}] --- Требования к составу документации \sysabbr{}
приведены в разделе \ref{sec:=000414=00043E=00043A=000443=00043C=000435=00043D=000442=000438=000440=00043E=000432=000430=00043D=000438=000435}. 
\end{description}
Проектирование, испытание и ввод в действие \sysabbr{} в части вопросов
информационной безопасности должны осуществляться по согласованию
и под контролем Главного управления безопасности и защиты информации
Банка России.

Организационно–распорядительная, проектная и эксплуатационная документация
на \sysabbr{} в части вопросов информационной безопасности, должна
согласовываться с Главным управлением безопасности и защиты информации
Банка России.

\smallsubpointК \sysabbr{} требования по защите от утечки информации
по техническим каналам, в том числе, по каналам побочных электромагнитных
излучений и наводок, не предъявляются.

\strong{\smallpoint\label{=000422=000440=000435=000431=00043E=000432=000430=00043D=000438=00044F-=00043A-=000437=000430=000449=000438=000442=000435}Требования
к защите информации от несанкционированного доступа и обеспечению
информационной безопасности \sysabbr{}}

\smallsubpointОбязанности по администрированию средств защиты и механизмов
защиты, реализующих требования по обеспечению информационной безопасности
\sysabbr{} в целом и, при необходимости, для отдельных подразделений
Банка России должны возлагаться приказами (распоряжениями) соответствующих
руководителей подразделений Банка России на администраторов информационной
безопасности \sysabbr{} и администраторов информационной безопасности
подразделений Банка России соответственно. 

Администраторы информационной безопасности подразделений Банка России
должны назначаться из числа сотрудников подразделений, чьи сотрудники
являются пользователями \sysabbr{}, (далее --- эксплуатирующие подразделения)
с соответствующим изменением их должностных обязанностей. 

Администраторы информационной безопасности подразделений Банка России
должны осуществлять свою деятельность на основании:
\begin{itemize}
\item Положения об администраторе информационной безопасности подразделения
Банка России (Приложение 1 к Положению 316--П);
\item Положения об администраторе информационной безопасности подразделения
Банка России, назначенного для администрирования средств и механизмов
защиты, реализующих требования по обеспечению информационной безопасности
ИСПДн Банка России (Приложение 3 к Временному порядку обеспечения
безопасности персональных данных, обрабатываемых в информационных
системах персональных данных Банка России).
\end{itemize}


На администраторов информационной безопасности подразделений Банка
России руководством этих подразделений могут возлагаться дополнительные
обязанности по администрированию механизмов защиты иной информации,
которая по мнению руководства указанных подразделений, подлежит защите. 

Администратор информационной безопасности \sysabbr{} (далее --- администратор
информационной безопасности системы) должен назначаться из числа сотрудников
обслуживающего подразделения (Центр информационных технологий Банка
России -- для центрального аппарата, подразделения информатизации
-- для других подразделений Банка России) с соответствующим изменением
его должностных обязанностей. 

По согласованию с Главным управлением безопасности и защиты информации
Банка России администратор информационной безопасности системы может
быть назначен не из числа сотрудников обслуживающего подразделения
(Центр информационных технологий Банка России --- для центрального
аппарата, подразделения информатизации --- для других подразделений
Банка России). 

Администратор информационной безопасности системы должен действовать
в соответствии с инструкцией, входящей в состав эксплуатационной документации
\sysabbr{}. 

Для реализации требований данного пункта должны быть разработаны:
\begin{itemize}
\item руководство администратора ИБ центрального звена (см. таблицу \ref{tab:Documents});
\item руководство администратора ИБ регионального звена (см. таблицу \ref{tab:Documents}).
\end{itemize}


\smallsubpointФункции администратора информационной безопасности
системы и администратора системы \sysabbr{} (далее --- администратор
системы) должны быть разделены организационными мерами и техническими
средствами. 

\smallsubpointАдминистратор системы не должен иметь полномочий и
технических средств по настройке параметров, влияющих на полномочия
пользователей по доступу к информации в данной системе. 

\smallsubpointАдминистратор системы не должен иметь полномочий и
технических средств по настройке параметров, влияющих на полномочия
пользователей по доступу к информации в данной системе. 

\smallsubpointАдминистратор системы должен иметь право включить в
число пользователей системы нового пользователя без каких--либо полномочий
по доступу к защищаемой информации, а также исключить из системы такого
пользователя.

\smallsubpointАдминистратор информационной безопасности системы должен
иметь полномочия и технические возможности по контролю действий соответствующих
администраторов системы без вмешательства в их действия и пользователей,
а также полномочия и технические средства по настройке для каждого
пользователя \sysabbr{} только тех параметров, которые определяют
права его доступа к информации в системе. \smallsubpointАдминистратор
информационной безопасности системы не должен иметь права включить
в число пользователей \sysabbr{} нового пользователя, а также удалить
из нее существующего пользователя.

\smallsubpointАдминистратор информационной безопасности подразделения
Банка России в пределах совокупности помещений, программно--технических
средств, а также сотрудников, работающих в указанных помещениях и
с указанными средствами, соблюдение требований по обеспечению информационной
безопасности для которых входит в его компетенцию и ответственность
(далее — зона ответственности) должен иметь полномочия и соответствующие
технические средства для контроля соответствия действий пользователей
\sysabbr{} и администратора системы установленному регламенту эксплуатации
\sysabbr{}. Контроль может осуществляться техническими средствами
либо с помощью организационных мер и может предусматривать как последовательные,
так и одновременные (параллельные) действия администратора информационной
безопасности системы и (или) администратора информационной безопасности
подразделения Банка России по отношению к действиям администратора
системы и ее пользователей. 

\smallsubpointАдминистратор информационной безопасности подразделения
Банка России в зоне своей ответственности может выполнять отдельные
функции администратора информационной безопасности системы. Такое
перераспределение полномочий должно быть отражено разработчиком \sysabbr{}
в эксплуатационной документации. 

\smallsubpointДопускается назначение одного лица администратором
информационной безопасности нескольких подразделений Банка России,
относящихся к одной сфере деятельности. Допускается назначение одного
лица администратором информационной безопасности \sysabbr{} и администратором
информационной безопасности других АС Банка России, а также совмещение
указанной функции с обязанностями администратора информационной безопасности
подразделения Банка России. Совмещение в одном лице функций администратора
информационной безопасности подразделения Банка России или администратора
информационной безопасности системы и функций администратора системы
не допускается. \smallsubpointНе допускается участие в технологическом
процессе обработки защищаемой информации администратора информационной
безопасности подразделения Банка России, администратора информационной
безопасности системы, администратора базы данных и других сотрудников,
которым по роду работы были предоставлены полномочия по управлению
\sysabbr{}, при использовании ими своих полномочий в качестве пользователя.
Администратор информационной безопасности подразделения Банка России,
администратор информационной безопасности системы, администратор системы,
администратор базы данных и другие сотрудники, которым по роду работы
были предоставлены полномочия по управлению \sysabbr{}, не должны
иметь полномочий по вводу и выводу информации, обрабатываемой \sysabbr{},
за исключением действий, определенных эксплуатационной документацией
(см. таблицу \ref{tab:Documents}). 

\smallsubpointРуководители эксплуатирующих и обслуживающих подразделений
Банка России должны обеспечивать безопасность защищаемой информации
в процессе эксплуатации \sysabbr{}. Сотрудники, осуществляющие обработку
и (или) хранение защищаемой информации должны соблюдать требования
нормативных и иных актов в области информационной безопасности. Для
реализации данного пункта требований должен быть описан процесс деятельности
сотрудников эксплуатирующих и обслуживающих подразделений в руководствах
администратора ИБ (см. таблицу \ref{tab:Documents}, документы И6,
И6.02). 

\smallsubpointСредства защиты информации от несанкционированного
доступа (далее --- НСД), используемые в составе ПИБ \sysabbr{}, должны
иметь подтвержденный сертификатом класс защиты не ниже четвертого
(в соответствии с руководящим документом Гостехкомиссии России «Средства
вычислительной техники. Защита от НСД к информации. Показатели защищенности
от НСД к информации») или класс защиты не ниже 3А, 2А или 1В (в соответствии
с руководящим документом Гостехкомиссии России «Автоматизированные
системы. Защита от НСД к информации. Классификация автоматизированных
систем и требования по защите информации»). 

\smallsubpointМежсетевые экраны должны иметь подтвержденный сертификатом
класс защиты не ниже четвертого при возможности информационного обмена
между всеми компонентами \sysabbr{} без использования компонентов
других АС (в иных случаях --- не ниже третьего класса). Указанные
классы защиты устанавливаются в соответствии с руководящим документом
Гостехкомиссии России «Средства вычислительной техники. Межсетевые
экраны. Защита от несанкционированного доступа к информации. Показатели
защищенности от несанкционированного доступа к информации». 

\smallsubpointИспользование сертифицированных средств защиты информации
от НСД и межсетевых экранов с закончившимся сроком действия сертификата
допускается в том случае, если указанные средства были приобретены
в период действия сертификата. В исключительных случаях и только по
согласованию с Главным управлением безопасности и защиты информации
Банка России возможно использование иных (несертифицированных) средств
защиты от НСД. 

\smallsubpointПроцессы подготовки, ввода, обработки, хранения и (или)
передачи защищаемой информации, а также порядок установки, настройки,
эксплуатации и восстановления необходимых технических и программных
средств должны быть регламентированы разработчиком \sysabbr{} в проектной
и эксплуатационной документации. 

\smallsubpointСостав и назначение ПО АРМ, в том числе СПО \sysabbr{},
должны быть зафиксированы в паспортах ПО АРМ. Порядок заполнения и
ведения паспортов ПО АРМ должен регламентироваться разработчиками
\sysabbr{} в эксплуатационной документации. Требования к форме паспортов
ПО АРМ, порядку их заполнения и ведения изложены в Приложении 2 к
Положению 316--П. 

\smallsubpointТехнические и программные средства, предназначенные
для разработки и отладки программного обеспечения либо содержащие
средства разработки, отладки и тестирования программно-аппаратного
обеспечения, должны располагаться в сегментах ЛВС, изолированных (на
уровне не выше сетевого в соответствии с эталонной моделью взаимодействия
открытых систем --- моделью OSI\nomenclature{OSI}{Open Systems Interconnect — взаимодействие открытых систем})
от сегментов, задействованных в подготовке, вводе, обработке, хранении
и (или) передаче защищаемой информации. Параметры настроек технических
и программных средств, обеспечивающих указанное разделение, а также
процедура контроля этих параметров настроек должны быть регламентированы
разработчиком в эксплуатационной документации \sysabbr{}. Стандартные
программные средства общего назначения, получившие широкое распространение
в информационных технологиях Банка России (например, MS Office), которые
не обеспечивают возможность по выборочному удалению из них средств
разработки и отладки ПО, могут быть использованы в сегментах, задействованных
в подготовке, вводе, обработке, хранении и (или) передаче защищаемой
информации при условии, что введен запрет (организационными мерами,
инструкциями, регламентами) использования отдельных их компонент (средств
разработки и отладки ПО). 

\smallsubpointПорядок внесения изменений в установленное ПО АРМ \sysabbr{},
осуществляющей обработку и (или) хранение защищаемой информации, включая
контроль действий программистов в процессе модификации ПО, должен
быть регламентирован. Эталонные копии ПО должны быть учтены, доступ
к ним должен быть регламентирован. Соответствующие регламенты в виде
инструкций, руководств должны готовиться разработчиком \sysabbr{}
в эксплуатационной документации. При возникновении необходимости модификации
ПО \sysabbr{}, обрабатывающего защищаемую информацию, должна осуществляться
централизованная рассылка официально уполномоченным подразделением
или специализированной организацией модификаций ПО с контролем его
целостности и подтверждением подлинности принятыми в Банке России
методами по согласованному с соответствующим подразделением безопасности
и защиты информации регламенту. По согласованию с Департаментом информационных
систем и Главным управлением безопасности и защиты информации может
быть установлен иной способ рассылки модификаций ПО. 

\smallsubpointПорядок действий администратора информационной безопасности
подразделения Банка России, администратора информационной безопасности
системы и персонала, занятых в процессе обработки и (или) хранения
защищаемой информации, должен быть определен руководствами администратора
ИБ\nomenclature{ИБ}{Информационная безопасность}
(см. таблицу \ref{tab:Documents}, документы И6, И6.02). 

Указанные инструкции (руководства): 
\begin{itemize}
\item должны разрабатываться с учетом Положения Банка России от 26.05.2006
№ 287--П «Об обеспечении сохранности конфиденциальных сведений в Банке
России» (далее по тексту — Положение 287--П);
\item должны устанавливать требования к квалификации в области защиты информации
администратора информационной безопасности подразделения Банка России,
администратора информационной безопасности системы и персонала, а
также актуальный перечень защищаемых объектов;
\item должны содержать в полном объеме данные о полномочиях пользователей;
\item должны содержать данные о технологии обработки информации в объеме,
необходимом для администратора информационной безопасности подразделения
Банка России и администратора информационной безопасности системы;
\item должны содержать параметры конфигурации средств защиты и механизмов
защиты информации от НСД, используемых в зоне ответственности администратора
информационной безопасности подразделения Банка России и администратора
информационной безопасности системы, а также должны определять порядок
и периодичность проверок установленных параметров конфигурации, но
не реже чем раз в год;
\item должны устанавливать порядок и периодичность анализа журналов регистрации
событий (архивов журналов);
\item должны регламентировать другие действия администратора информационной
безопасности подразделения Банка России, администратора информационной
безопасности системы и персонала, предусмотренные настоящим ТЗ. 
\end{itemize}


\smallsubpointВосстановление ПИБ \sysabbr{} в случае нештатной ситуации
должно осуществляться администратором системы с обязательным привлечением
администратора информационной безопасности системы (при необходимости
--- с привлечением специалистов подразделений информатизации и (или)
безопасности и защиты информации, а также администратора (администраторов)
информационной безопасности подразделения (подразделений) Банка России,
эксплуатирующего (эксплуатирующих) \sysabbr{}). Процедура восстановления
ПИБ должна быть регламентирована разработчиком \sysabbr{} в эксплуатационной
документации. 

\smallsubpointВыполнение функций защиты информации в \sysabbr{}
должно обеспечиваться комплексом встроенных механизмов защиты электронных
вычислительных машин (ЭВМ\nomenclature{ЭВМ}{Электронная вычислительная машина}),
ОС, СУБД\nomenclature{СУБД}{Система управления базами данных},
прикладного ПО, а также сертифицированных средств защиты информации. 

В случае невозможности использования сертифицированных средств защиты
информации допускается, по согласованию с Главным управлением безопасности
и защиты информации Банка России, использование только перечисленных
в абзаце первом настоящего пункта встроенных механизмов защиты. 

При этом в обязательном порядке на стадии ввода в действие разработчиком
\sysabbr{} должны быть выполнены настройки перечисленных в абзаце
первом настоящего пункта механизмов защиты, не допускающие несанкционированного
изменения пользователем предоставленных ему полномочий. Также разработчиком
\sysabbr{} должен быть предложен, согласован с соответствующим подразделением
безопасности и защиты информации и отражен в утвержденных руководством
подразделения Банка России руководствах администратора ИБ (см. таблицу
\ref{tab:Documents}, документы И6, И6.02) порядок постоянного контроля
фактического состояния указанных настроек на предмет их соответствия
установленным правилам. 

\smallsubpointИдентификация и аутентификация (проверка подлинности)
субъекта доступа при входе в \sysabbr{} должна обеспечиваться по
идентификатору (коду) и периодически обновляемому паролю длиной не
менее шести буквенно--цифровых символов. 

При наличии технической возможности количество последовательных неудачных
попыток ввода пароля должно быть ограничено --- от 3 до 5 попыток.
При превышении указанного количества средства защиты и механизмы защиты
должны блокировать возможность дальнейшего ввода пароля, включая правильное
значение пароля, до вмешательства администратора информационной безопасности
подразделения Банка России или администратора информационной безопасности
системы. 

Порядок формирования и смены паролей, а также контроля исполнения
этих процедур должен регламентироваться разработчиком \sysabbr{}
в эксплуатационной документации в руководствах администратора ИБ в
соответствии с требованиями к организации парольной защиты (Приложение
3 к Положению 316--П). 

\smallsubpointИдентификация информационных ресурсов (например, информационных
массивов, баз данных, файлов, обрабатывающих их программ), содержащих
защищаемую информацию, должна осуществляться по логическим именам. 

\smallsubpointКонтроль доступа субъектов к защищаемым информационным
ресурсам в соответствии с правами доступа указанных субъектов должен
являться обязательным.

\smallsubpointИнформация производственного участка \sysabbr{} должна
обрабатываться, храниться и (или) передаваться в \sysabbr{}, как
защищаемая информация. Информация тестового участка \sysabbr{} является
тестовой информацией, специально генерируемой для выполнения тестирования
функций \sysabbr{}, и не должна обрабатываться, храниться и (или)
передаваться в \sysabbr{} как защищаемая информация.

\smallsubpointВ \sysabbr{} для разделения потоков информации ограниченного
доступа и персональных данных должно обеспечиваться управление потоками
информации, для чего должна быть предусмотрена автоматизированная
маркировка защищаемых информационных ресурсов, в том числе создаваемых,
соответствующей меткой конфиденциальности. 

При этом уровень конфиденциальности накопителей (машинных носителей)
должен быть не ниже максимального уровня конфиденциальности записываемой
на них информации. 

\smallsubpoint\label{=000420=000435=000433=000438=000441=000442=000440=000430=000446=000438=00044F-=000432=000445=00043E=000434=000430-=000432}Регистрация
входа в \sysabbr{} (выхода из \sysabbr{}) субъекта доступа должна
являться обязательной. 

В журнале регистрации событий, который ведется в электронном виде
в \sysabbr{}, должны указываться следующие параметры:
\begin{itemize}
\item дата и время входа в систему (выхода из системы) субъекта доступа; 
\item идентификатор субъекта, предъявленный при запросе доступа; 
\item результат попытки входа: успешная или неуспешная (несанкционированная);
\item идентификатор (адрес) устройства (компьютера), используемого для входа
в систему. 
\end{itemize}


\smallsubpointРегистрация печати материалов, содержащих защищаемую
информацию, должна являться обязательной.

В журнале регистрации событий, который ведется в электронном виде
в \sysabbr{}, должны указываться следующие параметры:
\begin{itemize}
\item дата и время печати;
\item спецификация устройства печати (логическое имя (номер) внешнего устройства);
\item информация об уровне конфиденциальности материала (ограничительная
пометка «Для служебного пользования»);
\item полное наименование (вид, шифр, код) материала;
\item идентификатор субъекта доступа, запросившего печать материала;
\item объем фактически отпечатанного материала (количество страниц, листов,
копий) и результат печати: успешная (весь объем) или неуспешная. 
\end{itemize}


Печать должна сопровождаться автоматической маркировкой реквизитами,
предусмотренными Положением 287--П. В уже существующих электронных
технологиях, в которых автоматическая маркировка не может быть реализована,
проставление ограничительной пометки «Для служебного пользования»
и иных реквизитов должно быть регламентировано соответствующим подразделением
информатизации по согласованию с подразделением безопасности и защиты
информации в эксплуатационной документации. 

\smallsubpointРегистрация запуска программ и процессов, осуществляющих
доступ к защищаемым информационным ресурсам, должна являться обязательной. 

В журнале регистрации событий, который ведется в электронном виде
в \sysabbr{}, должны указываться следующие параметры:
\begin{itemize}
\item дата и время запуска;
\item имя (идентификатор) программы (процесса, задания); 
\item идентификатор субъекта доступа, запросившего программу (процесс, задание); 
\item результат попытки запуска: успешная или неуспешная (несанкционированная); 
\item дата и время попытки доступа к защищаемому информационному ресурсу; 
\item имя (идентификатор) защищаемого информационного ресурса; 
\item вид запрашиваемой операции (например, чтение, запись, удаление); 
\item результат попытки доступа: успешная или неуспешная (несанкционированная). 
\end{itemize}


\smallsubpoint\label{=000420=000435=000433=000438=000441=000442=000440=000430=000446=000438=00044F-=000438=000437=00043C=000435=00043D=000435=00043D=000438=000439-=00043F=00043E=00043B=00043D=00043E=00043C=00043E=000447=000438=000439}Регистрация
изменений полномочий субъектов доступа и статуса объектов доступа
(защищаемых информационных ресурсов) должна являться обязательной. 

В журнале регистрации событий, который ведется в электронном виде
\sysabbr{}, должны указываться следующие параметры:
\begin{itemize}
\item дата и время изменения;
\item содержание изменения с указанием идентификатора субъекта доступа,
чьи полномочия подверглись изменению, или логического имени защищаемого
информационного ресурса, чей статус изменился;
\item идентификатор администратора информационной безопасности подразделения
Банка России или администратора информационной безопасности системы,
осуществившего изменение.
\end{itemize}


\smallsubpointВ \sysabbr{} не должно быть субъекта доступа, имеющего
полномочия, а при возможности, и технические средства по уничтожению
и модификации информации, содержащейся в журналах регистрации событий,
указанных в пунктах \ref{sub:=000422=000440=000435=000431=00043E=000432=000430=00043D=000438=00044F-=00043A-=000437=000430=000449=000438=000442=000435}.\ref{=000422=000440=000435=000431=00043E=000432=000430=00043D=000438=00044F-=00043A-=000437=000430=000449=000438=000442=000435}.\ref{=000420=000435=000433=000438=000441=000442=000440=000430=000446=000438=00044F-=000432=000445=00043E=000434=000430-=000432}
-- \ref{sub:=000422=000440=000435=000431=00043E=000432=000430=00043D=000438=00044F-=00043A-=000437=000430=000449=000438=000442=000435}.\ref{=000422=000440=000435=000431=00043E=000432=000430=00043D=000438=00044F-=00043A-=000437=000430=000449=000438=000442=000435}.\ref{=000420=000435=000433=000438=000441=000442=000440=000430=000446=000438=00044F-=000438=000437=00043C=000435=00043D=000435=00043D=000438=000439-=00043F=00043E=00043B=00043D=00043E=00043C=00043E=000447=000438=000439}.

Очистка журналов регистрации событий должна регламентироваться разработчиком
\sysabbr{} в эксплуатационной документации. Перед очисткой журналов
регистрации событий должно производиться архивирование содержащейся
в них информации, путем перемещения информации в соответствующий архив.

Операция по архивированию журнала регистрации событий должна, в свою
очередь, регистрироваться с указанием времени и идентификатора сотрудника,
выполнившего операцию, в качестве первой записи в действующем журнале
регистрации событий. 

Архивы журналов регистрации событий должны уничтожаться только администратором
информационной безопасности подразделения Банка России или администратором
информационной безопасности системы, в зоне ответственности которого
находятся данные архивы через три года с момента появления последней
записи в данной архивной копии. 

\smallsubpointАвтоматизированный учет создаваемых в \sysabbr{} защищаемых
информационных ресурсов должен являться обязательным с целью их включения
в систему контроля доступа. 

\smallsubpointПри наличии технической возможности должна осуществляться
принудительная очистка (перезапись произвольной константой) освобождаемых
областей памяти на магнитных и иных носителях, ранее использованных
для хранения защищаемых информационных ресурсов. 

Постановка на учет машинных носителей, предназначенных для размещения
защищаемой информации, должна производиться в соответствии с Положением
287--П. 

Снятие с учета машинных носителей, на которых была размещена защищаемая
информация, должна производиться по акту путем стирания на них информации
средствами гарантированного стирания информации или по акту путем
их уничтожения в соответствии с Положением 287--П. Оба акта — по форме,
приведенной в Приложении 2 Положения 287--П.

Процедура стирания информации должна регламентироваться разработчиком
\sysabbr{} в эксплуатационной документации в зависимости от применяемого
средства гарантированного стирания. 

\smallsubpointСохранность и целостность программных средств ПИБ \sysabbr{},
защищаемой информации, а также других программных средств \sysabbr{}
должна являться обязательной и должна обеспечиваться в том числе,
за счет создания резервных копий. Резервному копированию должны подлежать
все программные средства, архивы, журналы, информационные ресурсы
(данные), используемые и создаваемые в процессе эксплуатации \sysabbr{}.

Порядок создания и сопровождения резервных копий, включающий способ
и периодичность копирования, процедуры создания, учета, хранения,
использования (для восстановления) и уничтожения резервных копий,
должны регламентироваться разработчиком \sysabbr{} в эксплуатационной
документации. 

\smallsubpointПользователи и обслуживающий персонал \sysabbr{} не
должны осуществлять несанкционированное и (или) нерегистрируемое (бесконтрольное)
копирование защищаемой информации. С этой целью в помещениях, в которых
размещаются технические средства обработки защищаемой информации,
должно запрещаться осуществление несанкционированного копирования,
в том числе, с использованием:
\begin{itemize}
\item отчуждаемых носителей информации, мобильных устройств копирования
и переноса информации, а также устройств фото и видеосъемки, коммуникационных
портов и устройств ввода--вывода;
\item инфракрасных портов и средств беспроводной связи мобильных устройств
(например, ноутбуков, карманных персональных компьютеров, смартфонов,
мобильных телефонов). 
\end{itemize}


\smallsubpointСредства восстановления функций ПИБ \sysabbr{} должны
предусматривать ведение не менее двух независимых копий программных
средств ПИБ \sysabbr{}. 

\smallsubpoint\label{=00041F=00043E=000440=00044F=000434=00043E=00043A-=000434=00043E=000441=000442=000443=00043F=000430-=000432 =00043F=00043E=00043C=000435=000449=000435=00043D=000438=000435}Порядок
доступа в помещение, в котором размещаются технические средства \sysabbr{},
должен регламентироваться подразделением безопасности и защиты информации,
в зону ответственности которого входит \sysabbr{}. 

\smallsubpointДля обеспечения защиты указанного в пункте 4.1.6.2.\ref{=00041F=00043E=000440=00044F=000434=00043E=00043A-=000434=00043E=000441=000442=000443=00043F=000430-=000432 =00043F=00043E=00043C=000435=000449=000435=00043D=000438=000435}
помещения на входной двери должен быть установлен замок, обеспечивающий
надежную защиту помещения в нерабочее время. 

При необходимости могут использоваться технические средства контроля
доступа в помещение. 

Необходимость сдачи помещения под охрану в нерабочее время должна
определяться руководителем эксплуатирующего подразделения Банка России
по согласованию с подразделением безопасности и защиты информации,
в зону ответственности которого входит \sysabbr{}. 

\smallsubpointДоступ к техническим средствам, задействованным в обработке,
хранении и (или) передаче защищаемой информации, их коммуникационным
портам и устройствам ввода--вывода информации, должен быть организован
с использованием программных и (или) программно--технических средств
блокировки и контроля доступа или с использованием специальных защитных
знаков, распространяемых централизованно Главным управлением безопасности
и защиты информации Банка России или закупаемых самостоятельно подразделениями
Банка России по согласованию с Главным управлением безопасности и
защиты информации Банка России.

Порядок организации доступа к техническим средствам, их коммуникационным
портам и устройствам ввода--вывода информации, а также порядок работы
со специальными защитными знаками, их учет и контроль целостности
должны соответствовать требованиям Положения Банка России от 17.02.2010
№ 355--П «Об организации доступа к коммуникационных портам и встроенным
устройствам ввода--вывода информации на технических средствах подразделений
Банка России». 

\smallsubpointПередача защищаемой информации между подразделениями
Банка России по телекоммуникационным каналам и линиям связи, не принадлежащим
Банку России или не пролегающим только по территории Банка России,
должна осуществляться только при обеспечении ее защиты с помощью виртуальных
частных сетей (Virtual Private Network — VPN\nomenclature{VPN}{Virtual Private Network — виртуальная  частная сеть}),
САЭД\nomenclature{САЭД}{Специализированный архиватор электронных документов}
или иных защитных мер, механизмов и средств, применение которых согласовывается
с Главным управлением безопасности и защиты информации Банка России.

\smallsubpointПередача защищаемой информации по телекоммуникационным
каналам и линиям связи между подразделениями Банка России, с одной
стороны, и внешними организациями производиться не должна. 

\smallsubpointЗащита информации от вредоносного кода, соответствующая
установленному в Банке России порядку, должна обеспечиваться в соответствии
с Положением Банка России от 01.11.2008 № 327--П «Об организации защиты
информационно--вычислительных ресурсов Банка России от воздействий
вредоносного кода».

\smallsubpointРегулярный контроль доступа пользователей к защищаемым
информационным ресурсам на соответствие установленным правилам доступа
должен быть регламентирован в руководствах администратора ИБ (см.
таблицу \ref{tab:Documents}, документы И6, И6.02).

\smallsubpointВсе полномочия по доступу к защищаемым информационным
ресурсам в \sysabbr{} должны быть персональными, указываться явно
и проверяться перед предоставлением доступа. По запросу на доступ
к информационным ресурсам должны предоставляться полномочия, минимально
необходимые для реализации данного запроса.

\strong{\smallpoint\label{=000422=000440=000435=000431=00043E=000432=000430=00043D=000438=00044F-=00043A-=000437=000430=000449=000438=000442=000435-=00041F=000414=00043D}Требования
к защите персональных данных, обрабатываемых в \sysabbr{}}

\smallsubpointОбъем и содержание ПДн, обрабатываемых с целью, приведенной
в пункте 4.1.6.1.3 настоящего ТЗ, должны быть приведены в документе
«\sysabbr{}. Пояснительная записка к техническому проекту. Часть
2. Описание решений по ИБ» и должны соответствовать Уведомлению об
обработке ПДн, направленному Банком России в Роскомнадзор.

\smallsubpointДля обработки в \sysabbr{} персональных данных с целью,
приведенной в пункте 4.1.6.1.3 настоящего ТЗ, требуется получение
письменного согласия субъектов, чьи персональные данные обрабатываются
в \sysabbr{}.

\smallsubpointВ письменном согласии должны содержаться следующие
сведения:
\begin{itemize}
\item фамилия, имя, отчество, адрес субъекта ПДн, номер основного документа,
удостоверяющего его личность, сведения о дате выдачи указанного документа
и выдавшем его органе;
\item наименование и адрес Банка России (структурного подразделения Банка
России), получающего согласие субъекта ПДн;
\item цель обработки ПДн;
\item перечень ПДн, на обработку которых дается согласие субъекта ПДн;
\item перечень действий с ПДн, на совершение которых дается согласие, общее
описание используемых способов обработки ПДн;
\item срок, в течение которого действует согласие, а также порядок его отзыва. 
\end{itemize}


Примерная форма письменного согласия на обработку персональных данных
установлена в приложении 2 к Временному порядку обработки персональных
данных в информационных системах персональных данных Банка России
и может применяться, если нормативными и иными актами Банка России
не установлена иная форма такого согласия для конкретных случаев.

\smallsubpointПодразделения, эксплуатирующие \sysabbr{} обязаны
предоставить (Роскомнадзору или субъекту ПДн при его обращении) доказательство
получения согласия субъекта ПДн на обработку его ПДн (копию письменного
согласия).

\smallsubpointРуководителями эксплуатирующих \sysabbr{} структурных
подразделений Банка России (для центрального аппарата Банка России
- руководителем структурного подразделения центрального аппарата Банка
России) должен быть документально определен список лиц, доступ которых
к ПДн, обрабатываемым в \sysabbr{}, необходим для выполнения служебных
обязанностей.

\smallsubpointПДн при их обработке в \sysabbr{} должны быть обособлены
от иной информации, в частности, путем фиксации их на отдельных материальных
носителях ПДн, в специальных разделах или на полях форм (бланков)
(для обработки ПДн на бумажных носителях).

\smallsubpointВ процессе эксплуатации \sysabbr{} не осуществляется
трансграничная передача персональных данных, обрабатываемых в \sysabbr{}.


\subsection{\label{sub:=000422=000440=000435=000431=00043E=000432=000430=00043D=000438=00044F-=00043A-=000444=000443=00043D=00043A=000446=000438=00044F=00043C,}Требования
к функциям, выполняемым \sysabbr{}}


\subsubsection{\label{sub:=000422=000440=000435=000431=00043E=000432=000430=00043D=000438=00044F-=00043A-=000424=00041F1}Требования
к характеристикам и функциям ФП планирования аудиторских проверок }

\smallpointЗадачи ФП планирования аудиторских проверок должны обеспечивать
автоматизацию работы со следующими видами планов:
\begin{itemize}
\item план аудиторских проверок службы главного аудитора;
\item план аудиторских проверок подразделений Московского ГТУ Банка России;
\item планы аудиторских проверок центров внутреннего аудита;
\item планы аудиторских проверок территориальных учреждений Банка России
(далее --- ТУ);
\item план аудиторских проверок Департамента полевых учреждений;
\item сводный план аудиторских проверок полевых учреждений Банка России.\end{itemize}
\begin{description}
\item [{Примечание}] --- Для реализации в \sysabbr{} функций ведения и
контроля планов аудиторских проверок  ТУ Банка России должны быть
использованы функции ведения планов аудиторских проверок ТУ Банка
России, реализованные в АС МРП.
\end{description}
\smallpoint\label{=000417=000430=000434=000430=000447=000430-=000420=000438=000441=00043A-=00043E=000440=000438=000435=00043D=000442=000438=000440=00043E=000432=000430=00043D=00043D=00043E=000435-=00043F=00043B=000430=00043D}Задача
Риск-ориентированное планирование аудиторских проверок должна обеспечивать
автоматизацию формирования проектов планов аудиторских проверок на
основе:
\begin{itemize}
\item установленной для объектов аудита периодичности проверок;
\item информации об устранении нарушений и недостатков, выявленных на объектах
аудита;
\item информации о событиях, влияющих на деятельность объектов аудита.
\end{itemize}


\smallsubpointЗадача Риск-ориентированное планирование аудиторских
проверок должна обеспечивать автоматизацию следующих процедур формирования
проекта плана аудиторских проверок:
\begin{itemize}
\item формирование перечня аудиторских проверок, предлагаемых к включению
в план аудиторских проверок;
\item просмотр информации об аудиторских проверках, предлагаемых к включению
в план;
\item включение предлагаемых аудиторских проверок в проект плана аудиторских
проверок;
\item оценка выполнимости формируемого проекта плана аудиторских проверок
с учетом имеющихся в распоряжении подразделения службы главного аудитора
Банка России кадровых ресурсов;
\item редактирование формируемого проекта плана аудиторских проверок.
\end{itemize}


\smallsubpointСостав учитываемой при формировании перечня аудиторских
проверок, предлагаемых к включению в план, информации (перечисленной
в пп. \ref{sub:=000422=000440=000435=000431=00043E=000432=000430=00043D=000438=00044F-=00043A-=000424=00041F1}.\ref{=000417=000430=000434=000430=000447=000430-=000420=000438=000441=00043A-=00043E=000440=000438=000435=00043D=000442=000438=000440=00043E=000432=000430=00043D=00043D=00043E=000435-=00043F=00043B=000430=00043D})
должен выбираться исходя из режима формирования:
\begin{itemize}
\item автоматический --- на основе типа формируемого плана и прав доступа
пользователя к информации \sysabbr{};
\item настраиваемый --- на основе настроек, введенных пользователем.
\end{itemize}


\smallsubpointПри формировании перечня аудиторских проверок, предлагаемых
к включению в план, пользователю \sysabbr{} должна быть предоставлена
возможность включения и отключения групп условий, обеспечивающих учет:
\begin{itemize}
\item установленной для объектов аудита периодичности проверок;
\item информации об устранении нарушений и недостатков, выявленных на объектах
аудита;
\item планов аудиторских проверок за предыдущие годы;
\item информации о событиях влияющих на деятельность объектов аудита.
\end{itemize}






\smallsubpointПо каждой аудиторской проверке, предлагаемой к включению
в план, пользователю \sysabbr{} должна предоставляться следующая
информация:
\begin{itemize}
\item наименование объекта аудита;
\item вид предлагаемой проверки;
\item состав проверяемых направлений деятельности объекта аудита; 
\item предполагаемая трудоемкость аудиторской проверки;
\item предлагаемый проверяемый период;
\item предлагаемое время проведения;
\item причины включения в перечень предлагаемых аудиторских проверок.
\end{itemize}


\smallsubpointПри включении аудиторской проверки в проект плана аудиторских
проверок должно быть обеспечено формирование и сохранение в составе
плана текстового комментария с описанием причины включения проверки
в план. 

\smallsubpointТекстовый комментарий в зависимости от причины включения
проверки в план должен содержать:
\begin{itemize}
\item значение установленное для объекта аудита периодичности проведения
аудиторских проверок;
\item описание направлений деятельности или вида операций, в которых были
выявлены нарушения при проведении предыдущей проверки;
\item описание имевшего место на объекте аудита неблагоприятного события.
\end{itemize}


\smallsubpointПри редактировании проекта плана аудиторских проверок
должна быть обеспечена возможность: 
\begin{itemize}
\item редактирования информации о планируемой аудиторской проверке;
\item добавления и редактирование проверок, не вошедших в перечень предлагаемых
проверок;
\item объединения нескольких тематических аудиторских проверок в одну комплексную
или тематическую;
\item разделения аудиторской проверки на несколько.
\end{itemize}


\smallpointЗадача Ведение планов аудиторских проверок должна обеспечивать:
\begin{itemize}
\item сохранение проекта плана в качестве версии плана;
\item создание версии плана при импорте плана из внешних источников;
\item хранение планов аудиторских проверок;
\item ведение версий планов аудиторских проверок;
\item ведение истории изменения планов аудиторских проверок;
\item ведение истории изменения статусов планов аудиторских проверок; 
\item предоставление информации о различиях версий планов аудиторских проверок.
\end{itemize}


\smallpointЗадача Контроль и согласование планов аудиторских проверок
должна обеспечивать автоматизацию:
\begin{itemize}
\item формирования запросов ДВА о предоставлении предложений о проведении
аудиторских проверок;
\item формирования указаний ДВА на включение аудиторских проверок в планы
ЦВА;
\item контроля соответствия планов аудиторских проверок установленным требованиям
к периодичности их проведения;
\item контроля назначения аудиторских проверок по вопросам устранения выявленных
нарушений и недостатков;
\item контроля изменений планов аудиторских проверок;
\item формирования электронных документов с планами аудиторских проверок
и изменениями планов аудиторских проверок;
\item доведения планов аудиторских проверок и требований к ним до согласующих
и утверждающих лиц.
\end{itemize}


\smallpointТребования к задачам и функциям ФП планирования аудиторских
проверок должны быть приведены в документе «\sysabbr{}. Описание
постановки комплекса задач ФП планирования аудиторских проверок».


\subsubsection{\label{sub:=000422=000440=000435=000431=00043E=000432=000430=00043D=000438=00044F-=00043A-=000424=00041F2}Требования
к характеристикам и функциям ФП поддержки подготовки аудиторских проверок}



\smallpointЗадача Формирование пакета документов по аудиторской проверке
должна обеспечивать автоматизацию формирования и согласования пакета
документов по аудиторской проверке, включающего:
\begin{enumerate}
\item предписание на проведение аудиторской проверки;
\item предписание о внесении изменений в предписание на проведение аудиторской
проверки;
\item задание на проведение аудиторской проверки;
\item справку о подготовке аудиторской проверки;
\item телеграммы о привлечении сотрудников к аудиторской проверке;
\item распоряжения о командировании сотрудников; 
\item служебные задания для сотрудников; 
\item запросы в профильные подразделения для получения предложений по вопросам
для проверки; 
\item заявки на предоставление документов, доступ к базам данных объекта
аудита; 
\item командировочные удостоверения сотрудников ДВА;
\item типовой порядок работы с материалами проверки; 
\item типовой порядок оформления акта аудиторской проверки; 
\item другие распорядительные документы. 
\end{enumerate}


\smallsubpointФормирование документов должно осуществляться в соответствии
с шаблонами путем подстановки в шаблоны документов актуальной информации
о подготавливаемой аудиторской проверке. Шаблон документа может также
предусматривать необходимость ввода дополнительных параметров, которые
должны запрашиваться у пользователя при формировании документа в соответствии
с шаблоном.

\smallsubpointСостав формируемых документов, обеспечивающих проведение
аудиторской проверки, должен определяться составом шаблонов документов,
определенных для проверяемого объекта аудита, либо для соответствующего
типа объектов аудита. 

\smallsubpointАвтоматизация заполнения шаблонов при формировании
документов должна обеспечиваться в объеме доступной в \sysabbr{}
информации о подготавливаемой аудиторской проверке и объекте проверки.
Недостающие сведения должны вводиться пользователем в соответствующие
разделы шаблона вручную с использованием офисного пакета Microsoft
Office. 

\smallsubpointДля обеспечения автоматизированного формирования пакета
документов по аудиторской проверке должна быть реализована функция
ввода и корректировки данных по подготавливаемой аудиторской проверке. 

\smallsubpointВ состав информации по подготавливаемой аудиторской
проверке, вводимой пользователем \sysabbr{}, должны быть включены: 
\begin{itemize}
\item плановый период проведения аудиторской проверки;
\item объекты аудита; 
\item вид аудиторской проверки (комплексная / тематическая); 
\item признак плановости проведения проверки (плановая / внеплановая); 
\item номер пункта плана аудиторских проверок (для плановых проверок); 
\item проверяемые направления деятельности (для тематических проверок); 
\item сроки проведения проверки;
\item планируемый проверяемый период;
\item лицо, назначившее аудиторскую проверку
\item состав группы проверяющих. 
\end{itemize}


\smallsubpointДолжен быть обеспечен ввод в \sysabbr{} информации
о событии завершения подготовки аудиторской проверки.

\smallsubpointПри вводе и корректировке данных по подготавливаемой
аудиторской проверке должна быть предусмотрена возможность ввода следующей
информации о результатах подготовки аудиторской проверки:
\begin{itemize}
\item <<аудиторская проверка начата>>; 
\item <<аудиторская проверки отменена>>. 
\end{itemize}


\smallsubpointВ случае, если по результатам подготовки аудиторской
проверки начата аудиторская проверка, должен в автоматическом режиме
формироваться и помещаться в информационное хранилище \sysabbr{}
отчет об аудиторской проверке, содержащий сведения о начале аудиторской
проверки. 

\smallsubpointДолжно обеспечиваться хранение документов, формируемых
в процессе подготовки аудиторской проверки, отображение перечня хранимых
документов и возможность загрузки хранимых документов на АРМ пользователя
для просмотра и редактирования с использованием офисного пакета Microsoft
Office. 

\smallsubpointПри отображении перечня хранимых документов, обеспечивающих
проведение аудиторской проверки, должны отображаться следующие атрибуты
каждого документа:
\begin{enumerate}
\item вид документа, в соответствии со справочником видов документов; 
\item дополнительные атрибуты документа, в зависимости от вида документа
(например, Ф.И.О. сотрудника); 
\item имя файла документа. 
\end{enumerate}


\smallsubpointДолжна обеспечиваться загрузка на АРМ пользователей
как отдельных документов (преимущественно с целью просмотра), так
и загрузка комплекта документов с возможностью корректировки документов
в офисном пакете Microsoft Office и обратной загрузки скорректированных
редакций документов в архив материалов аудиторских проверок. 

\smallsubpointДолжно обеспечиваться ведение истории изменения хранимых
документов с возможностью просмотра списка версий и загрузки любой
сохраненной версии документа на АРМ пользователя.

\smallpointЗадача Формирование состава групп проверяющих должна включать
функции, обеспечивающие автоматизацию:
\begin{itemize}
\item отбора кандидатур для включения в состав группы проверяющих;
\item включения информации о сотрудниках, включенных в состав группы проверяющих
в справку о подготовке аудиторской проверки;
\item включения информации о сотрудниках в телеграммы о привлечении сотрудников
к аудиторской проверке. 
\end{itemize}


\smallsubpointПри автоматизированном отборе кандидатур для включения
в группу проверяющих должны учитываться: 
\begin{enumerate}
\item перечень подразделений СГА Банка России для отбора сотрудников;
\item перечень требуемых для аудиторской проверки специализаций сотрудников;
\item наличие опыта участия в аудиторских проверках определенного вида,
с возможностью указания:

\begin{itemize}
\item вида аудиторской проверки (для тематических – перечень проверяемых
направлений деятельности);
\item типа объекта аудита (в соответствии со справочником типов объектов
аудита);
\item признака участия в качестве руководителя проверки; 
\item оценок за участие в аудиторских проверках; 
\item минимального количества участий;
\end{itemize}
\item планы по участию сотрудника в других аудиторских проверках, планируемые
отпуска и периоды обучения сотрудника.
\end{enumerate}






\smallsubpointДолжна быть обеспечена возможность многократного выполнения
операций по изменению критериев отбора кандидатур и получению перечня
сотрудников, соответствующих введенным критериям.

\smallsubpointПользователю \sysabbr{} должны быть обеспечены возможности:
\begin{itemize}
\item исключения ранее выбранных сотрудников из состава группы проверяющих; 
\item корректировки специализации каждого выбранного сотрудника в рамках
планируемой аудиторской проверки.
\end{itemize}


\smallsubpointДетальная информация о сотрудниках, отображаемая средствами
\sysabbr{} при отборе кандидатур для включения в группу проверяющих
должна включать: 
\begin{itemize}
\item сведения о занятости сотрудника в период проведения планируемой аудиторской
проверки; 
\item сведения об участии сотрудника в ранее проведенных аудиторских проверках,
с возможностью просмотра детальной информации по каждой аудиторской
проверке; 
\item сводку по видам нарушений, выявленных сотрудником, с возможностью
просмотра детальной информации по отдельным выявленным нарушениям. 
\end{itemize}
\textbf{Примечание} --- задачи Формирование пакета документов по аудиторской
проверке и Формирование состава групп проверяющих должны быть реализованы
с использованием задач и функций, реализованных в рамках АС МРП.

\smallpointЗадача Подготовка справки по объекту аудита должна включать
функции получения и структурированного представления информации по
следующим разделам:
\begin{itemize}
\item описания направлений деятельности, осуществляемой объектом аудита;
\item информация о мероприятиях по устранению недостатков, выявленных на
объекте аудита;
\item информация о неблагоприятных событиях, влияющих на деятельность объекта
аудита.
\end{itemize}


\smallsubpointФункции задачи подготовки справки по объекту аудита
должны обеспечивать управление составом информации включаемой в справку
посредством выбора:
\begin{itemize}
\item состава разделов, включаемых в справку;
\item определения состава объектов аудита, по которым формируется справка;
\item определение для каждого из объектов аудита состава направлений деятельности,
планируемых к проверке.
\end{itemize}


\smallsubpointДолжно быть обеспечено формирование справки по объекту
аудита в следующих режимах:
\begin{itemize}
\item автоматический --- на остове введенной пользователем информации об
аудиторской проверке;
\item настраиваемый --- на основе настроек, введенных пользователем.
\end{itemize}


\smallsubpointДолжна быть обеспечена возможность корректировки формируемых
справок по объектам аудита в офисном пакете Microsoft Office и возможность
отчуждения справок по объектам аудита с АРМ пользователя на машиночитаемый
носитель.

\smallpointДетальные требования к задачам и функциям ФП поддержки
подготовки аудиторских проверок должны быть приведены в документе
«\sysabbr{}. Описание постановки комплекса задач ФП поддержки подготовки
аудиторских проверок».


\subsubsection{\label{sub:=000422=000440=000435=000431=00043E=000432=000430=00043D=000438=00044F-=00043A-=000424=00041F3}Требования
к характеристикам и функциям ФП поддержки реализации материалов аудиторской
проверки и контроля за устранением нарушений и недостатков}



\smallpointЗадача Ввода актов аудиторских проверок должна обеспечивать
автоматизацию ввода в \sysabbr{} актов аудиторских проверок. 

\smallsubpointФункциями задачи должен быть обеспечен ввод содержания
актов аудиторских проверок разделам акта.

\smallsubpointДолжно быть предусмотрено изменение состава и наименований
разделов акта аудиторской проверки.



\smallsubpointЗадача ввода актов аудиторских проверок должна обеспечивать
автоматизацию формирования выписок из актов аудиторских проверок для
отправки в профильные департаменты центрального аппарата Банка России.

\smallsubpointЗадача ввода актов аудиторских проверок должна обеспечивать:
\begin{itemize}
\item внесение изменений в хранимые акты аудиторских проверок и ввод решений
о внесении изменений в акт аудиторской проверки;
\item ведение истории корректировки актов аудиторских проверок. 
\end{itemize}


\smallpointЗадача Импорт планов мероприятий по устранению нарушений
и недостатков должна включать следующие функции:
\begin{itemize}
\item импорт планов мероприятий из файлов исходных данных в формате Microsoft
Office;
\item сохранение в структурированном виде импортированных планов мероприятий
в информационном хранилище \sysabbr{};
\item импорта скорректированных планов мероприятий из файлов исходных данных
в формате Microsoft Office;
\item просмотр импортированных планов мероприятий в виде документа формата
Microsoft Office;
\item ведение истории изменений планов мероприятий по устранению недостатков;
\item корректировки импортированных планов мероприятий;
\item управление версиями планов мероприятий.
\end{itemize}


\smallpointЗадача Импорт отчетов о выполнении плана мероприятий должна
включать следующие функции:
\begin{itemize}
\item импорт отчетов о выполнении плана мероприятий файлов исходных данных
в формате Microsoft Office;
\item сохранение в структурированном виде импортированных отчетов о выполнении
планов мероприятий в информационном хранилище \sysabbr{};
\item просмотр импортированных планов мероприятий в виде документа формата
Microsoft Office.
\end{itemize}


\smallpointЗадача Контроль выполнения планов мероприятий должна обеспечивать:
\begin{itemize}
\item отслеживание выполнения запланированных мероприятий по устранению
нарушений и недостатков;
\item сохранение информации о событиях завершения мероприятий по устранению
нарушений и недостатков; 
\item формирование отчетов по выполнению планов мероприятий;
\item ввод информации о необходимых аудиторских проверках по вопросам устранения
недостатков.
\end{itemize}


\smallsubpointДолжно быть обеспечено формирование отчетов о выполнении
планов мероприятий для:
\begin{itemize}
\item отдельного объекта аудита; 
\item отдельной аудиторской проверки;
\item группы объектов аудита;
\item набора аудиторских проверок.
\end{itemize}


\smallsubpointЗадача Контроль выполнения планов мероприятий по устранению
недостатков должна обеспечивать:
\begin{itemize}
\item просмотр и сохранение формируемых отчетов в виде документа формата
Microsoft Office;
\item отчуждение сформированных отчетов с АРМ пользователя \sysabbr{} на
машиночитаемый носитель.
\end{itemize}


\smallpointТребования к задачам и функциям ФП поддержки подготовки
решений и контроля устранения недостатков должны быть приведены в
документе «\sysabbr{}. Описание постановки комплекса задач ФП поддержки
подготовки решений и контроля устранения недостатков».


\subsubsection{\label{sub:=000422=000440=000435=000431=00043E=000432=000430=00043D=000438=00044F-=00043A-=000424=00041F4}Требования
к характеристикам и функциям ФП мониторинга аудиторских проверок}

\smallpointЗадача Ввод и передача в центральное звено информации
об аудиторских проверках должна обеспечивать:
\begin{itemize}
\item ввод в информационное хранилище \sysabbr{} планов аудиторских проверок;
\item ввод в информационное хранилище \sysabbr{} отчетов о проведении аудиторских
проверок;
\item импорт и ввод в информационное хранилище \sysabbr{} отчетности из
файлов в формате Microsoft Office;
\item импорт и ввод в информационное хранилище \sysabbr{} планов аудиторских
проверок из файлов в формате Microsoft Office. 
\end{itemize}




\smallpointЗадача Формирование и анализ сводного плана аудиторских
проверок должна обеспечивать формирование сводного плана аудиторских
проверок на основе планов аудиторских проверок подразделений СГА.



\smallpointЗадача Анализ выполнения планов аудиторских проверок должна
обеспечивать формирование отчетов о выполнении:
\begin{itemize}
\item планов аудиторских проверок подразделений СГА;
\item групп планов аудиторских проверок группы подразделений СГА;
\item сводного плана аудиторских проверок. 
\end{itemize}


\smallpointЗадача Анализ динамики выявления нарушений должна обеспечивать
автоматизацию статистического анализа изменения количества выявляемых
нарушений.



\smallpointЗадача Мониторинг состояния аудиторских проверок должна
включать функции:
\begin{itemize}
\item отображения состояния аудиторских проверок;
\item отображения состояния отдельной аудиторской проверки.
\end{itemize}


\smallsubpointФункция отображения состояния аудиторской проверки
должна обеспечивать отображение сводной информации об аудиторских
проверках за заданный период, включая:
\begin{itemize}
\item наименование объекта  аудита; 
\item плановый период проведения аудиторской проверки; 
\item фактический период проведения аудиторской проверки (при наличии, на
основании поступившего отчета); 
\item состояние аудиторской проверки (планируется, начата, завершена).
\end{itemize}


\smallsubpointФункция отображения состояния аудиторской проверки
должна обеспечивать отображение детальной информации о состоянии аудиторской
проверки, включая:
\begin{itemize}
\item этап работ по проверке; 
\item информацию о составе документов, сформированных по проверке;
\item информацию о составе введенных материалов аудиторских проверок;
\item наличии/отсутствии введенного акта аудиторской проверки.
\end{itemize}


\smallsubpointДолжна быть обеспечена возможность просмотра (при наличии
соответствующих прав доступа) доступных на текущий момент времени:
\begin{itemize}
\item сформированных документов по аудиторской проверке;
\item введенных материалов аудиторской проверки;
\item акта аудиторской проверки;
\item плана мероприятий;
\item отчетов о выполнении планов мероприятий.
\end{itemize}


\smallsubpointЗадача Мониторинг состояния аудиторских проверок должна
обеспечивать регистрацию значимых событий по аудиторским проверкам.
Для каждого значимого события подготовки аудиторских проверок должны
сохраняться следующие параметры: 
\begin{itemize}
\item время события; 
\item имя пользователя \sysabbr{}, выполнившего действие, связанное с событием; 
\item вид события;
\item дополнительные параметры, в зависимости от вида события. 
\end{itemize}
\textbf{Примечание} --- задачи ФП мониторинга аудиторских проверок
должны быть реализованы с использованием задач и функций, реализованных
в рамках АС МРП.

\smallpointТребования к задачам ФП мониторинга аудиторских проверок,
должны быть приведены в документе «\sysabbr{}. Описание постановки
комплекса задач ФП мониторинга аудиторских проверок».


\subsubsection{\label{sub:=000422=000440=000435=000431=00043E=000432=000430=00043D=000438=00044F-=00043A-=000424=00041F5}Требования
к характеристикам и функциям ФП внутреннего документооборота СГА}

\smallpointФП внутреннего документооборота СГА должна обеспечивать
автоматизацию внутреннего документооборота СГА, возникающего при управлении
подразделениями СГА и не связанного непосредственно с выполнением
процедур внутреннего аудита.

\smallpointФункции ФП внутреннего документооборота СГА должны обеспечивать
выполнение установленных цепочек обработки документов, включающих
следующие этапы обработки:
\begin{itemize}
\item регистрация документа;
\item согласование документа; 
\item визирование документа;
\item исполнение документа.
\end{itemize}


\smallpointДля каждого из видов документов, обрабатываемых в рамках
внутреннего документооборота СГА должны устанавливаться правила обработки,
определяющие:
\begin{itemize}
\item состав, последовательность и нормативную продолжительность этапов
обработки документа;
\item состав ролей участников обработки документов.
\end{itemize}


\smallpointФункции ФП внутреннего документооборота СГА должны обеспечивать
возможность:
\begin{itemize}
\item учета документов;
\item отслеживания обработки документов;
\item получения информации о количестве документов, находящихся в обработке
у каждого из участников;
\item формирования участникам обработки документов уведомлений об установленных
сроках обработки документов.
\end{itemize}


\smallpointТребования к функциям ФП внутреннего документооборота
СГА, а также к составу видов документов, обрабатываемых в рамках ФП
внутреннего документооборота СГА должны быть приведены в документе
<<\sysabbr{}. Описание постановки комплекса задач ФП внутреннего
документооборота СГА>>.


\subsubsection{\label{sub:=000422=000440=000435=000431=00043E=000432=000430=00043D=000438=00044F-=00043A-=000424=00041F6}Требования
к характеристикам и функциям ФП информационных ресурсов СГА}



\smallpointЗадача Ведение данных о квалификации и занятости сотрудников
СГА должна включать функции, обеспечивающие: 
\begin{itemize}
\item ввод и корректировку информации о периодах занятости сотрудников и
причинах занятости (отпуска, больничные, обучение, участие в аудиторских
проверках);
\item ввод и корректировку информации о квалификации сотрудников;
\item ведение истории изменения информации о квалификации сотрудников;
\item импорт информации о квалификации и занятости сотрудников из файлов
в формате Microsoft Office.
\end{itemize}


\smallsubpointЗадача Ведение данных о квалификации и занятости сотрудников
СГА должна включать следующие функции предоставления информации о
сотрудниках СГА:
\begin{enumerate}
\item печать перечня сотрудников подразделения СГА;
\item печать детальной информации о сотруднике СГА; 
\item отображение и экспорт в формат Microsoft Office перечня сотрудников
подразделения СГА; 
\item поиск сотрудника СГА по любой комбинации полей:

\begin{itemize}
\item Ф.И.О.;
\item название должности;
\item номер телефона;
\item подразделение; 
\end{itemize}
\item отображение и экспорт в офисный формат Microsoft Office детальной
информации о сотруднике СГА.
\end{enumerate}


\smallpointЗадача Ведение шаблонов документов должна включать функции,
обеспечивающие ввод и корректировку шаблонов документов, формируемых
при выполнении:
\begin{itemize}
\item планирования аудиторских проверок;
\item подготовки аудиторских проверок;
\item проведения аудиторских проверок; 
\item реализации материалов аудиторских проверок;
\item контроля устранения недостатков, выявленных аудиторскими проверками. 
\end{itemize}


\smallpointЗадача Ведение справочников должна включать функции, обеспечивающие
ведение справочников и классификаторов \sysabbr{}, необходимых для
формирования информационно--аналитического обеспечения деятельности
СГА. 

\textbf{Примечание} --- Задачи Ведение данных о квалификации и занятости
сотрудников СГА, Ведение шаблонов документов, Ведение справочников
должны быть реализованы с использованием задач и функций разработанных
в рамках АС МРП.

\smallpointЗадача Ведение реестра аудиторских проверок должна включать
функции, обеспечивающие ведение единого реестра аудиторских проверок. 

\smallsubpointДолжен быть обеспечен ввод в единый реестр аудиторских
проверок следующей информации по аудиторским проверкам:
\begin{itemize}
\item состоянии аудиторских проверок;
\item значимых событиях по аудиторской проверке на всех стадиях - от планирования
аудиторской проверки до завершения контроля мероприятий по устранению
выявленных нарушений;
\item информации о сформированных при подготовке к аудиторской проверке
документах;
\item сведений о введенных материалах аудиторской проверки.
\end{itemize}


\smallsubpointДолжна быть обеспечена возможность ввода пользователем
следующих сведений реестра аудиторских проверок:
\begin{itemize}
\item информации о введенных материалах аудиторских проверок;
\item информации о событиях по аудиторской проверке;
\item информации о изменении состояния аудиторской проверки.
\end{itemize}


\smallsubpointРегламент ведения реестра аудиторских проверок должен
быть определен на стадии технорабочего проекта первой очереди \sysabbr{}.

\smallsubpointДолжна быть обеспечена возможность поиска аудиторских
проверок в реестре аудиторских проверок по комбинациям полей:
\begin{itemize}
\item время начала подготовки аудиторской проверки; 
\item время начала проведения аудиторской проверки; 
\item время ввода акта аудиторской проверки;
\item состояние (статус) аудиторской проверки;
\item вид аудиторской проверки;
\item объект аудита. 
\end{itemize}


\smallpointЗадача Ведение архива материалов аудиторских проверок
должна включать функции, обеспечивающие: 
\begin{itemize}
\item хранение документов по аудиторской проверке, сформированных на стадии
подготовки;
\item хранение введенных материалов аудиторских проверок;
\item хранение структурированной актов аудиторских проверок; 
\item планов мероприятий;
\item отчетов о выполнении планов мероприятий; 
\item управленческих решений;
\item решений о внесении изменений в акты аудиторских проверок;
\item ведение истории изменения актов аудиторских проверок.
\end{itemize}


\smallpointЗадача Ведение каталога объектов аудита должна обеспечивать
ведение и предоставление следующей информации по объектам:
\begin{itemize}
\item информации о направлениях деятельности и операциях, осуществляемых
объектом аудита;
\item информации о событиях, влияющих на деятельность объектов аудита; 
\item информации об устранении выявленных нарушений и недостатков на объектах
аудита.
\end{itemize}


\smallpointДетальные требования к задачам и функциям ФП ведения реестра
информационных ресурсов СГА, должны быть приведены в документе «\sysabbr{}.
Описание постановки комплекса задач ФП реестр информационных ресурсов
СГА».


\subsubsection{Требования к характеристикам и функциям ОП информационного хранилища}

\smallpointОП информационного хранилища СГА должна обеспечивать хранение:
\begin{itemize}
\item информации основных справочников системы и истории корректировки справочников;
\item перечня объектов аудита, описаний объектов аудита и истории изменения
описаний объектов аудита;
\item знаний по объектам аудита;
\item планов аудиторских проверок с поддержкой истории изменения планов;
\item данных о ходе и результатах аудиторских проверок;
\item данных о мероприятиях по устранению выявленных нарушений и недостатков;
\item шаблонов документов, используемых при подготовке и проведении аудиторских
проверок, и истории изменения шаблонов. 
\end{itemize}


\smallpointДля каждой аудиторской проверки должно обеспечиваться
хранение сведений:
\begin{itemize}
\item объект, вид, сроки проведения проверки; 
\item состав группы проверяющих; 
\item журнал хода проверки;
\item исходные материалы проверки;
\item состав и структура выявленных нарушений;
\item информация о принятом управленческом решении;
\item план мероприятий по устранению нарушений и недостатков;
\item отчеты об устранении выявленных нарушений и недостатков;
\item состав проверок по контролю устранения нарушений и недостатков.
\end{itemize}


\smallpointДля каждого мероприятия по устранению нарушений и недостатков
должно обеспечиваться хранение следующих сведений:
\begin{itemize}
\item перечень устраняемых нарушений и недостатков;
\item описание мероприятий;
\item период проведения мероприятий;
\item ответственные сотрудники;
\item информация об устранении нарушений и недостатков по результатам мероприятий.
\end{itemize}


\smallpointОП информационного хранилища СГА должна обеспечивать синхронизацию
хранимой информации между центральной базой данных и региональными
серверами кэширования данных.


\subsubsection{Требования к характеристикам и функциям ОП рабочего документооборота}

\smallpointОП рабочего документооборота должна обеспечивать обмен
электронными документами (далее --- ЭД\nomenclature{ЭД}{Электронный документ})
между участниками процессов, автоматизируемых \sysabbr{}.

\smallpointВ состав функций ОП рабочего документооборота должны включаться:
\begin{itemize}
\item регистрация рабочих ЭД СГА;
\item контроль соответствия работы с ЭД различных типов установленным для
типа ЭД правилам обработки;
\item учет событий обработки рабочих ЭД СГА;
\item поиск ЭД;
\item ведение архива ЭД.
\end{itemize}


\smallpointРеализуемые ОП рабочего документооборота функции контроля
соответствия работы с ЭД установленным правилам должны обеспечивать:
\begin{itemize}
\item контроль корректности последовательности операций по обработке ЭД
установленной последовательности обработки;
\item контроль своевременности обработки ЭД участниками обработки ЭД;
\item формирование в соответствии с установленными правилами информационных
сообщений и предупреждений участникам обработки ЭД.
\end{itemize}


\smallpointОП рабочего документооборота должна обеспечивать учет
следующих событий обработки ЭД:
\begin{itemize}
\item начало и завершение этапов обработки ЭД;
\item корректировка содержимого ЭД;
\item изменение статуса ЭД;
\item пересылка ЭД;
\item изменение участника процесса, ответственного за обработку ЭД.
\end{itemize}

\subsubsection{Требования к характеристикам и функциям ОП внутрисистемного взаимодействия}

\smallpointОП внутрисистемного взаимодействия должна обеспечивать:
\begin{itemize}
\item механизм синхронизации данных централизованного информационного хранилища
и серверов кэширования данных;
\item обмен защищенными транспортными сообщениями между пользовательскими
АРМ и информационным хранилищем в процессе исполнения функций системы;
\item защищенный доступ к приложениям \sysabbr{} с использованием «тонкого»
веб--клиента с рабочих мест пользователей.
\end{itemize}

\subsubsection{Требования к характеристикам и функциям ОП администрирования}

\smallpointОП администрирования должна обеспечивать:
\begin{itemize}
\item автоматическое распространение основных справочников \sysabbr{};
\item централизованное ведения учетных записей пользователей \sysabbr{};
\item ведение протоколов выполнения технологических операций \sysabbr{};
\item протоколирование сбоев при функционировании прикладных компонентов
\sysabbr{};
\item контроль состояния основных компонентов \sysabbr{}; 
\item фиксацию отклонений от штатного состояния и информирование административного
персонала;
\item автоматизированное распространение обновлений прикладного программного
обеспечения \sysabbr{}.
\end{itemize}

\subsubsection{Требования к характеристикам и функциям ПИБ \sysabbr{}}

\smallpointПри создании ПИБ \sysabbr{} должны быть выработаны и
реализованы решения по усилению контроля за действиями эксплуатирующего
персонала \sysabbr{} в ЦИТ со стороны АИБ в ДВА, предусматривающие: 
\begin{enumerate}
\item создание защищенного хранилища событий ИБ; 
\item обеспечение сбора и оперативной передачи событий ИБ, возникающих на
уровне ОС и СУБД серверов \sysabbr{}, в защищенное хранилище; 
\item обеспечение доступа для АИБ \sysabbr{} в ДВА к данным о событиях
ИБ, помещенным в защищенное хранилище.
\end{enumerate}


\smallpointЗащищенное хранилище событий ИБ и средства приема данных
о событиях ИБ должны быть размещены на серверных мощностях, подконтрольных
ДВА. Средства сбора и передачи данных о событиях ИБ из журналов ОС
и СУБД должны быть размещены на серверах производственного участка
\sysabbr{}. 

\smallpointДолжен быть обеспечен сбор данных о следующих видах событий
ИБ из протоколов ОС и СУБД: 
\begin{enumerate}
\item входы администраторов в ОС и СУБД на серверах \sysabbr{}; 
\item изменение привилегий, установленных на уровне ОС и СУБД; 
\item доступ к защищаемым информационным ресурсам (файлам, таблицам СУБД)
от имени привилегированных учетных записей ОС и СУБД. 
\end{enumerate}


\smallpointФиксация данных о событиях ИБ в защищенном хранилище должна
осуществляться в оперативном режиме, при условии доступности средств
приема данных о событиях ИБ. 

\smallpointПрямое взаимодействие средств сбора данных о событиях
ИБ, размещенных на основном и резервном серверах информационного хранилища
производственного участка, со средствами приема данных о событиях
ИБ, не допускается. 

Передача данных о событиях ИБ в защищенное хранилище должна выполняться
через основной и резервный серверы доступа.

\smallpointФункции сбора, оперативной обработки, консолидации, хранения
событий ИБ системно-технического уровня, а также предоставления доступа
к событиям ИБ для АИБ \sysabbr{} в ДВА должны быть реализованы на
базе промышленного комплекса средств мониторинга событий ИБ – продукта
IBM Tivoli Security Information and Event Manager.

\smallpointДля обеспечения функционирования программных средств,
реализующих функции приёма, хранения и предоставления пользователям
данных о событиях ИБ, должен быть выделен дополнительный сервер информационной
безопасности \sysabbr{}, размещаемый на производственных площадях
Заказчика по адресу: г. Москва, Ленинский пр-т, д.1 к.1. 

\textbf{Примечание} --- Требования к серверу информационной безопасности
\sysabbr{} приведены в подразделе \ref{sub:=000422=000440=000435=000431=00043E=000432=000430=00043D=000438=00044F-=00043A-=000442=000435=000445=00041E=000431=000435=000441=00043F=000435=000447=000435=00043D=000438=00044E}.


\subsection{\label{sub:=000422=000440=000435=000431=00043E=000432=000430=00043D=000438=00044F-=00043A-=00041E=000431=000435=000441=00043F}Требования
к видам обеспечения}


\subsubsection{Требования к информационному обеспечению}

\smallpointИнформационное взаимодействие между серверными компонентами
центрального звена \sysabbr{}, серверами кэширования данных и автономными
АРМ пользователей \sysabbr{} должно быть организовано по принципу
обмена сообщениями. Форматы сообщений, используемые для обмена, должны
быть основаны на стандарте XML 1.0. 

\smallpointХранение данных \sysabbr{} средствами ОП информационного
хранилища должно быть реализовано в рамках центральной БД \sysabbr{},
функционирующей под управлением СУБД Oracle Database. Структура центральной
БД \sysabbr{} должна быть описана в документе «Описание информационного
обеспечения». 

\smallpointРешения по применению общероссийских классификаторов и
ведомственных справочников должны быть приняты в ходе технического
проектирования и описаны в документе «Описание информационного обеспечения» 

\smallpointХранение данных на автономных АРМ пользователей \sysabbr{}
должно быть реализовано с использованием встраиваемой реляционной
СУБД SQLite. Структура локальных баз данных, размещаемых на автономных
АРМ, должна быть определена в документе «Описание информационного
обеспечения». 

\smallpointФорматы XML\nomenclature{XML}{Extensible Markup Language — расширяемый язык разметки текста}--документов,
размещаемых в полях центральной БД \sysabbr{} и локальных баз данных
автономных АРМ, должны быть описаны в документе «Описание информационного
обеспечения».

\smallpointВ \sysabbr{} должно быть реализовано защищенное хранилище
данных о событиях информационной безопасности.

\smallpointВ документацию \sysabbr{} должно быть включено описание
базы данных защищенного хранилища данных о событиях информационной
безопасности.

\smallpointДолжен быть разработан и включен в документацию \sysabbr{}
перечень видов контролируемых со стороны АИБ \sysabbr{} в ДВА событий
информационной безопасности уровня ОС и СУБД.


\subsubsection{Требования к лингвистическому обеспечению}

\smallpointВызов операций в \sysabbr{} должен осуществляться с использованием
пунктов меню и экранных кнопок. Текст пунктов меню и экранных кнопок
должен согласовываться с Заказчиком в процессе разработки \sysabbr{}. 

\smallpointВыполнение заданных пользователями операций в \sysabbr{}
должно сопровождаться текстовыми экранными сообщениями. Сообщения
должны выводиться на русском языке, за исключением стандартных системных
сообщений, выдаваемых лицензионными программными средствами.


\subsubsection{Требования к программному обеспечению}

\smallpointПрикладное программное обеспечение (ППО\nomenclature{ППО}{Прикладное программное обеспечение}),
необходимое для реализации функций ФП \sysabbr{}, должно быть реализовано
в виде следующих программных комплексов (ПК\nomenclature{ПК}{Программный комплекс}),
которые могут создаваться по отдельным ТЗ:
\begin{itemize}
\item ФП планирования аудиторских проверок --- ПК планирования аудиторских
проверок;
\item ФП поддержки подготовки аудиторских проверок --- ПК подготовки аудиторских
проверок;
\item ФП поддержки проведения аудиторских проверок --- ПК поддержки проведения
аудиторских проверок и ПК АРМ <<Аудитор>>;
\item ФП поддержки подготовки решений и контроля устранения недостатков
--- ПК поддержки реализации результатов аудиторских проверок;
\item ФП мониторинга аудиторских проверок --- ПК Мониторинга аудиторских
проверок; 
\item ФП реестр информационных ресурсов СГА --- ПК реестра информационных
ресурсов СГА.
\end{itemize}


\smallpointДолжно быть разработано (ППО), необходимое для реализации
функций следующих ОП \sysabbr{}:
\begin{itemize}
\item ОП информационного хранилища;
\item ОП внутрисистемного взаимодействия;
\item ОП взаимодействия с внешними системами;
\item ОП администрирования. 
\end{itemize}


\smallpointЦентральная БД \sysabbr{} должна быть реализована на
основе промышленной реляционной СУБД Oracle Database, версии 10g.

\smallpointВсе подсистемы \sysabbr{}, за исключением автономных
АРМ и информационного хранилища, должны быть реализованы на программной
платформе Java2 Enterprise Edition (J2EE) с использованием сервера
приложений IBM WebSphere Application Server версии 6.1. 

\smallpointВ состав стандартного программного обеспечения \sysabbr{}
должны быть включены программные средства IBM Security Information
and Event Manager (IBM SIEM), реализующие функции мониторинга событий
ИБ системно-технического уровня.

\smallpointАрхитектура программного обеспечения \sysabbr{} должна
предусматривать исполнение функций по взаимодействию с СДС Банка России,
системой электронной почты Банка России и рабочими местами пользователей
центрального звена \sysabbr{} на выделенном сервере (сервере доступа),
на котором должно быть исключено хранение информации ограниченного
доступа.


\subsubsection{\label{sub:=000422=000440=000435=000431=00043E=000432=000430=00043D=000438=00044F-=00043A-=000442=000435=000445=00041E=000431=000435=000441=00043F=000435=000447=000435=00043D=000438=00044E}Требования
к техническому обеспечению}

\smallpointРазвертывание системы должно осуществляться с использованием
существующего КТС\nomenclature{КТС}{Комплекс технических средств}
производственного участка АС МРП, включая следующие компоненты КТС:
\begin{itemize}
\item сервер информационного хранилища;
\item сервер доступа;
\item станция поддержки шифрования данных;
\item устройство резервного копирования на магнитную ленту. 
\end{itemize}


\smallpointСервер информационного хранилища \sysabbr{} должен обеспечивать
функционирование центральной БД \sysabbr{} и необходимых средств
администрирования СУБД, а также должен обеспечивать размещение серверных
компонентов \sysabbr{} в среде IBM WebSphere Application Server. 

\smallpointСервер доступа центрального звена \sysabbr{} должен обеспечивать
функционирование основных серверных компонентов ОП внутрисистемного
взаимодействия, а также СКЗИ «Форт», используемого для обеспечения
защищенного доступа пользователей к приложениям \sysabbr{}. 

\smallpointСервер информационного хранилища \sysabbr{}, сервер доступа
центрального звена \sysabbr{} и устройство резервного копирования
на магнитную ленту должны размещаться на производственных площадях
ЦИТ. 

\smallpointДля реализации резервирования производственного участка
\sysabbr{} в состав КТС \sysabbr{} должны быть включены два дополнительных
сервера: 
\begin{itemize}
\item резервный сервер информационного хранилища \sysabbr{};
\item резервный сервер доступа центрального звена \sysabbr{}. 
\end{itemize}


\smallpointКонфигурация резервных серверов \sysabbr{} должна обеспечивать
возможность выполнения функций производственного участка в полном
объеме на период ремонта и/или перенастройки основных серверов \sysabbr{}. 

\smallpointДополнительное оборудование, необходимое для реализации
резервирования серверов \sysabbr{}, а также необходимые дополнительные
лицензии на системное программное обеспечение (ОС Windows Server,
СУБД Oracle Database) должны быть изысканы из собственных резервов
либо приобретены Заказчиком. 

\smallpointВ связи с ожидаемым увеличением требований по емкости
информационного хранилища, вызванным планируемой реализацией функций
ведения архива материалов аудиторских проверок и общим расширением
номенклатуры хранимых данных, емкость информационного хранилища \sysabbr{}
должна быть увеличена путем включения в состав производственного участка
\sysabbr{} двух внешних дисковых массивов (основного и резервного),
подключаемых к основному и резервному серверам информационного хранилища
производственного участка \sysabbr{}. 

\smallpointВнешние дисковые массивы должны обеспечить хранение не
менее 1~Тбайт данных каждый с обеспечением избыточности хранения
в режиме RAID-10 при использовании не менее четырех физических дисков. 

\smallpointВнешние дисковые массивы должны быть совместимы с основным
и резервным серверами информационного хранилища производственного
участка \sysabbr{} на уровне аппаратного обеспечения и операционной
системы. 

\smallpointДополнительное оборудование, необходимое для расширения
емкости информационного хранилища \sysabbr{}, должно быть изыскано
из собственных резервов либо приобретено Заказчиком. 

\smallpointВ состав КТС \sysabbr{} должен быть включен дополнительный
сервер информационной безопасности, на котором должны размещаться
серверные компоненты продукта IBM Tivoli Security Information and
Event Manager. Сервер информационной безопасности \sysabbr{} должен
соответствовать рекомендуемым требованиям, приведенным в документации
на IBM Tivoli Security Information and Event Manager. 

\smallpointДополнительное оборудование и лицензии на программное
обеспечение (ОС Microsoft Windows Server, ПО IBM Tivoli Security Information
and Event Manager), необходимые для реализации функций мониторинга
событий информационной безопасности на системно-техническом уровне,
должны быть изыскано из собственных резервов либо приобретены Заказчиком.

\smallpointТестирование компонентов \sysabbr{} осуществляется с
использованием тестового участка, создаваемого с использованием существующего
КТС тестового участка \sysabbr{}, включающего:
\begin{itemize}
\item тестовый сервер информационного хранилища \sysabbr{};
\item тестовый сервер доступа \sysabbr{};
\item тестовый совмещенный АРМ. 
\end{itemize}


\smallpointДля развертывания пользовательских АРМ \sysabbr{} должны
использоваться рабочие станции, на которых развернуты АРМ \sysabbr{},
либо вновь закупаемые или уже существующие рабочие станции, дополнительно
оснащаемые СЗИ. 

\smallpointРешение по применению уже существующих либо вновь закупаемых
рабочих станций для развертывания АРМ \sysabbr{} должно приниматься
Заказчиком.


\subsubsection{Требования к организационному обеспечению}

\smallpointВ рамках работ по созданию \sysabbr{} должны быть разработаны
предложения по организационному обеспечению \sysabbr{}.

\smallpointВ рамках работ по созданию \sysabbr{} должны быть разработаны
эксплуатационные документы, содержащие перечень функций персонала
\sysabbr{} и описание порядка взаимодействия персонала \sysabbr{}.

\newpage{}


\section{Состав и содержание работ по созданию \sysabbr{}}

\label{sec:=000421=00043E=000441=000442=000430=000432}

\bigpointРаботы по созданию \sysabbr{} выполняются в три очереди.

\bigpointСтадии создания первой очереди \sysabbr{} в настоящем ТЗ
определены в соответствии с ГОСТ~34.601--90.

\bigpointСтадии создания второй и третьей очередей \sysabbr{} должны
определяться в рамках работ по созданию соответствующих очередей.

\bigpointСодержание работ по созданию первой очереди \sysabbr{}
и сроки их проведения приведены в таблице~\ref{tab:Stages}.

\bigpointСодержание работ и сроки их проведения могут быть скорректированы
на этапе разработки договоров.

\newcounter{stage}
\newcommand{\stage}{\stepcounter{stage}\arabic{stage}\quad{}}
\newcounter{substage}[stage] \newcommand{\substage}{\stepcounter{substage}\arabic{stage}.\arabic{substage}\quad{}}

\begin{longtable}{|>{\raggedright}p{0.52\textwidth}|>{\centering}p{0.15\textwidth}|>{\raggedright}p{0.25\textwidth}|}
\caption{\label{tab:Stages}Состав и содержание работ}
\tabularnewline
\hline 
\noindent \centering{}Наименование работ & \noindent \centering{}Срок выполнения работ & \noindent \centering{}Форма завершения работ\tabularnewline
\hline 
\endfirsthead
\hline 
\noindent \centering{}Наименование работ & \noindent \centering{}Срок выполнения работ & \noindent \centering{}Форма завершения работ\tabularnewline
\hline 
\endhead
\emph{\stageПервый этап создания }\sysabbr{}\emph{. Технический
проект первой очереди }\sysabbr{} &  & \tabularnewline
\hline 
\substageРазработка технического проекта \sysabbr{} & до 15.11.2012 & Согласованные с заказчиком проекты документов технического проекта:
«Описание постановки комплекса задач ФП планирования аудиторских проверок»;\tabularnewline
 &  & «Описание постановки комплекса задач ФП поддержки подготовки аудиторских
проверок»;\tabularnewline
 &  & «Описание постановки комплекса задач ФП поддержки подготовки решений
и контроля устранения недостатков»;\tabularnewline
 &  & «Описание постановки комплекса задач ФП мониторинга аудиторских проверок»;\tabularnewline
 &  & «Описание постановки комплекса задач ФП внутреннего документооборота
СГА»;\tabularnewline
 &  & «Описание постановки комплекса задач ФП реестр информационных ресурсов
СГА»;\tabularnewline
 &  & «Описание комплекса технических средств»;«Ведомость технического проекта»; \tabularnewline
 &  & «Схема автоматизации»; «Пояснительная записка к техническому проекту.
Часть 1. Решения по реализации функций \sysabbr{}»; \tabularnewline
 &  & «Пояснительная записка к техническому проекту. Часть 2. Описание решений
по ИБ»\tabularnewline
 &  & «Описание информационного обеспечения»;\tabularnewline
 &  & «Описание программного обеспечения»; \tabularnewline
 &  & «Модель нарушителей и угроз»\tabularnewline
 &  & Акт завершения работ по этапу\tabularnewline
\hline 
\emph{\stage Второй этап создания} \sysabbr{}\emph{. Разработка
рабочего проекта первой очереди \sysabbr{} и ввод системы в действие } &  & \tabularnewline
\hline 
\substageРазработка рабочей эксплуатационной документации & до 28.06.2013 & Разработанная рабочая и эксплуатационная документация: «Схема деления»;\tabularnewline
 &  & «Спецификация»;\tabularnewline
 &  & «Ведомость эксплуатационной документации»;\tabularnewline
 &  & «Формуляр»;\tabularnewline
 &  & «Спецификация оборудования и программного обеспечения»;\tabularnewline
 &  & «Общее описание системы»;\tabularnewline
 &  & «Описание технологического процесса»;\tabularnewline
 &  & «Руководство пользователя»;\tabularnewline
 &  & «Руководство пользователя. АРМ администратора системы»;\tabularnewline
 &  & «Руководство администратора системы»;\tabularnewline
 &  & «Инструкция по эксплуатации комплекса технических средств»;\tabularnewline
 &  & «Руководство пользователя. АРМ администратора ИБ»;\tabularnewline
 &  & «Руководство администратора ИБ»\tabularnewline
\cline{1-1} \cline{3-3} 
Разработка ПО обеспечивающих подсистем \sysabbr{} &  & ПО обеспечивающих подсистем \sysabbr{}, на машиночитаемом носителе\tabularnewline
Дооснащение и настройка КТС СИП СГА  &  & Скомплексированный КТС \sysabbr{}\tabularnewline
Интеграция в \sysabbr{} задач и функций, разработанных в рамках АС~МРП &  & ПО \sysabbr{}, развернутое на КТС \sysabbr{} (в объеме прикладных
функций АС МРП)\tabularnewline
\substageРазработка ПО ФП СИП СГА &  & Разработанное ПО \sysabbr{} на МН\tabularnewline
\cline{1-1} \cline{3-3} 
\substageРазвертывание ПО первой очереди \sysabbr{} на КТС Заказчика &  & Программное обеспечение \sysabbr{}, развернутое на КТС Заказчика\tabularnewline
\cline{1-1} \cline{3-3} 
\substageРазработка приемо--сдаточной документации &  & Утвержденные документы: «Программа и методика испытаний»; \tabularnewline
 &  & «Программа опытной эксплуатации»\tabularnewline
\cline{1-1} \cline{3-3} 
\substageПредварительные испытания &  & Протокол предварительных комплексных испытаний \sysabbr{}. Акт о
вводе \sysabbr{} в опытную эксплуатацию\tabularnewline
\cline{1-1} \cline{3-3} 
\substageОпытная эксплуатация \sysabbr{} &  & Журнал опытной эксплуатации. Акт о завершении опытной эксплуатации\tabularnewline
\cline{1-1} \cline{3-3} 
\substageПриемочные испытания и ввод первой очереди \sysabbr{} в
постоянную эксплуатацию \sysabbr{} &  & Протокол приемочных испытаний. Акт приемки в постоянную эксплуатацию\tabularnewline
\hline 
\end{longtable}

\newpage{}


\section{Порядок контроля и приемки \sysabbr{} }

\label{sec:=00041A=00043E=00043D=000442=000440=00043E=00043B=00044C}


\subsection{Виды испытаний \sysabbr{}}

\sysabbr{} должна пройти следующие виды испытаний:
\begin{itemize}
\item предварительные испытания. Состав, объем и методы предварительных
испытаний каждой очереди \sysabbr{} определяются в отдельном документе
«Программа и методика испытаний»;
\item опытная эксплуатация. Состав, объем опытной эксплуатации каждой очереди
\sysabbr{} определяются в отдельном документе «Программа опытной
эксплуатации»;
\item приемочные испытания. Состав, объем и методы приемочных испытаний
каждой очереди \sysabbr{} определяются в отдельном документе «Программа
и методика испытаний».
\end{itemize}

\subsection{Предварительные испытания}

\pointПредварительные испытания \sysabbr{} проводят с целью определения
ее работоспособности, соответствия Техническому заданию и решения
вопроса о возможности приемки очереди \sysabbr{} в опытную эксплуатацию.
Предварительные испытания проводят путем выполнения тестов (контрольных
примеров), которые должны обеспечить проверку выполнения требований
к очереди \sysabbr{}, установленных в техническом задании.

\pointПредварительные испытания проводятся в соответствии с документом
«Программа и методика испытаний» на технических средствах Заказчика
(по согласованию с Заказчиком предварительные испытания могут проводиться
на технических средствах Исполнителя), в сроки, установленные Распоряжением
по Банку России, комиссией, в состав которой должны входить представители
следующих подразделений Банка России: ДВА, ДИС, ГУБиЗИ\nomenclature{ГУБиЗИ}{Главное управление безопасности и защиты информации},
ЦИТ\nomenclature{ЦИТ}{Центр информационных технологий}.
К работе комиссии должны быть привлечены представители Исполнителя.
Комиссию по проведению предварительных испытаний возглавляет представитель
Заказчика.

\pointРезультаты предварительных испытаний оформляются протоколом
испытаний. По итогам предварительных испытаний составляется Акт приемки
автоматизированной системы в опытную эксплуатацию, который подписывается
членами комиссии и утверждается лицом, определенным в Распоряжении
по Банку России.


\subsection{Опытная эксплуатация }

\pointОпытная эксплуатация проводится на основании акта приемки \sysabbr{}
в опытную эксплуатацию.

\pointОпытная эксплуатация проводится с целью проверки правильности
функционирования очереди \sysabbr{} при выполнении каждой функции
и готовности персонала к работе в условиях функционирования очереди
\sysabbr{}.

\pointПеречень подразделений Банка России, участвующих в опытной
эксплуатации (далее Подразделения), и сроки опытной эксплуатации определяются
Заказчиком и указываются в Акте приемки очереди \sysabbr{} в опытную
эксплуатацию (при необходимости в Распоряжении по Банку России о начале
опытной эксплуатации).

\pointОпытная эксплуатация \sysabbr{} проводится участниками опытной
эксплуатации в Подразделении на полном объеме реальных данных в соответствии
с программой опытной эксплуатации.

\pointПрограмма опытной эксплуатации должна содержать:
\begin{itemize}
\item условия и порядок функционирования \sysabbr{} (ее части);
\item продолжительность опытной эксплуатации;
\item порядок устранения недостатков, выявленных в процессе опытной эксплуатации.
\end{itemize}
\pointВ программу опытной эксплуатации отдельным разделом должны
включаться проверки выполнения требований Технического задания по
защите информации. Ответственными исполнителями за выполнение данного
раздела являются все участники опытной эксплуатации, в состав которых
в обязательном порядке должны входить представители УБиЗИ\nomenclature{УБиЗИ}{Управление безопасности и защиты информации}
и подразделений информатизации. Программа опытной эксплуатации должна
согласовываться с ГУБиЗИ.

\pointВо время опытной эксплуатации в каждом Подразделении ведется
журнал опытной эксплуатации, в который заносят сведения о функционировании,
отказах, сбоях, аварийных ситуациях, обеспечении требований к защите
информации.

\pointПо результатам опытной эксплуатации в каждом Подразделении
принимается решение о готовности (или неготовности) очереди \sysabbr{}
к приемке в постоянную эксплуатацию. Решение оформляется Актом о завершении
опытной эксплуатации. Акт о завершении опытной эксплуатации подписывается
участниками опытной эксплуатации, утверждается руководителем Подразделения
и направляется представителю Заказчика для дальнейшего его представления
на приемочные испытания.


\subsection{Приемочные испытания}

\pointОснованием для проведения приемочных испытаний является Акт
о завершении опытной эксплуатации и Распоряжение по Банку России о
создании комиссии по проведению приемочных испытаний.

\pointПриемочные испытания \sysabbr{} проводят с целью определения
соответствия системы Техническому заданию, анализа результатов опытной
эксплуатации и решения вопроса о возможности приемки очереди \sysabbr{}
в постоянную эксплуатацию.

\pointПриемочные испытания проводятся на реальных данных с учетом
анализа результатов опытной эксплуатации.

\pointПриемочные испытания проводятся в соответствии с документом
«Программа и методика испытаний» на технических средствах Заказчика.

\pointПриемочные испытания проводятся в установленные Распоряжением
по Банку России сроки, комиссией, в состав которой должны входить
представители следующих подразделений Банка России: ДВА, ДИС, ГУБиЗИ,
ЦИТ. К работе комиссии должны быть привлечены представители Исполнителя. 

\pointКомиссию по проведению приемочных испытаний возглавляет представитель
Заказчика.

\pointРезультаты приемочных испытаний оформляются протоколом испытаний.
По итогам приемочных испытаний комиссия:
\begin{itemize}
\item принимает решение о приемке \sysabbr{} в постоянную эксплуатацию
в подразделениях Заказчика, в случае положительных результатов приемочных
испытаний;
\item принимает решение о приемке \sysabbr{} в постоянную эксплуатацию
в тех Подразделениях, в которых проводилась опытная эксплуатация и
от которых были получены Акты о завершении опытной эксплуатации с
выводом о готовности к приемке \sysabbr{} в постоянную эксплуатацию;
\item рекомендует применять (или не применять) проектные решения, используемые
в \sysabbr{}, как типовые проектные решения при проведении работ
по вводу в эксплуатацию \sysabbr{} в системе Банка России. 
\end{itemize}
\pointРешение Комиссии оформляется Актом приемки \sysabbr{} в постоянную
эксплуатацию. Акт подписывается членами комиссии и утверждается лицом,
определенным в Распоряжении по Банку России.

\newpage{}


\section{Требования к документированию}

\label{sec:=000414=00043E=00043A=000443=00043C=000435=00043D=000442=000438=000440=00043E=000432=000430=00043D=000438=000435}

\bigpointДокументация \sysabbr{} должна быть разработана и оформлена
согласно требованиям ГОСТ~2.105--95, ГОСТ~34.201--89, РД~50--34.698-90,
ГОСТ~19.603--78.

\bigpointПеречень отчетных документов, предъявляемых Заказчику по
окончании работы, представлен в таблице~\ref{tab:Documents}.

\begin{longtable}{|>{\raggedright}p{0.52\textwidth}|>{\centering}p{0.15\textwidth}|>{\raggedright}p{0.25\textwidth}|}
\caption{\label{tab:Documents}Перечень отчётных документов \sysabbr{}}
\tabularnewline
\hline 
\noindent \centering{}Наименование отчетного документа & \noindent \centering{}Код вида документа & \noindent \centering{}Согласующие подразделения\tabularnewline
\endfirsthead
\hline 
\noindent \centering{}Наименование отчетного документа & \noindent \centering{}Код вида документа & \noindent \centering{}Согласующие подразделения\tabularnewline
\endhead
\hline 
\multicolumn{3}{|l|}{\emph{Проектная документация} \emph{\sysabbr{}}}\tabularnewline
\hline 
«\sysabbr{}. Ведомость технического проекта» & ТП & ДИС\tabularnewline
\hline 
«\sysabbr{}.Пояснительная записка к техническому проекту. Часть 1.
Основные технические решения» & П2 &  ДИС\tabularnewline
\hline 
«\sysabbr{}. Пояснительная записка к техническому проекту. Часть
2. Решения по обеспечению ИБ» & П2--2 &  ДИС, ГУБиЗИ\tabularnewline
\hline 
«\sysabbr{}. Описание постановки комплекса задач ФП планирования
аудиторских проверок» & П4 & ДВА\tabularnewline
\hline 
«\sysabbr{}. Описание постановки комплекса задач ФП поддержки подготовки
аудиторских проверок» & П4.02 & ДВА\tabularnewline
\hline 
«\sysabbr{}. Описание постановки комплекса задач ФП поддержки подготовки
решений и контроля устранения недостатков» & П4.03 & ДВА\tabularnewline
\hline 
«\sysabbr{}. Описание постановки комплекса задач ФП мониторинга аудиторских
проверок»; & П4.04 & ДВА\tabularnewline
\hline 
«\sysabbr{}. Описание постановки комплекса задач ФП реестр информационных
ресурсов СГА» & П4.05 & ДВА\tabularnewline
\hline 
«\sysabbr{}. Описание постановки комплекса задач ФП внутреннего документооборота
СГА» & П4.06 & ДВА\tabularnewline
\hline 
«\sysabbr{}. Описание информационного обеспечения»  & П5 &  ДИС\tabularnewline
\hline 
«\sysabbr{}. Ведомость покупных изделий» & ВП &  ДИС\tabularnewline
\hline 
«\sysabbr{}. Схема автоматизации» & С3 &  ДИС\tabularnewline
\hline 
«\sysabbr{}. Модель нарушителей и угроз»  & ПЕ &  ДИС, ГУБиЗИ\tabularnewline
\hline 
\multicolumn{3}{|l|}{\emph{Рабочая документация \sysabbr{}}}\tabularnewline
\hline 
«\sysabbr{}. Схема деления»  & Е1  & ДИС\tabularnewline
\hline 
«\sysabbr{}. Спецификация»  & —  & ДИС \tabularnewline
\hline 
\multicolumn{3}{|l|}{\emph{Эксплуатационная документация \sysabbr{}}}\tabularnewline
\hline 
«\sysabbr{}. Ведомость эксплуатационной документации» & ЭД &  ДИС \tabularnewline
\hline 
«\sysabbr{}. Формуляр»  & ФО  & ДИС\tabularnewline
\hline 
«\sysabbr{}. Спецификация оборудования и программного обеспечения
»  & В4 &  ДИС\tabularnewline
\hline 
«\sysabbr{}. Общее описание системы »  & ПД  & ДИС, ГУБиЗИ\tabularnewline
\hline 
«\sysabbr{}. Описание технологического процесса» & ПГ  & ДИС, ДВА, ГУБиЗИ\tabularnewline
\hline 
«\sysabbr{}. Руководство пользователя»  & И3  & ДИС, ДВА\tabularnewline
\hline 
«\sysabbr{}. Руководство пользователя. АРМ администратора системы»  & И3.02  & ДИС, ДВА\tabularnewline
\hline 
«\sysabbr{}. Руководство пользователя. АРМ администратора ИБ»  & И3.03  & ДИС, ДВА, ГУБиЗИ \tabularnewline
\hline 
«\sysabbr{}. Руководство администратора системы» & И5  & ДИС, ДВА\tabularnewline
\hline 
«\sysabbr{}. Руководство администратора системы в региональном звене
(типовое)»  & И5.02 &  ДИС, ДВА\tabularnewline
\hline 
«\sysabbr{}. Руководство администратора ИБ»  & И6  & ГУБиЗИ, ДВА\tabularnewline
\hline 
«\sysabbr{}. Руководство администратора ИБ в региональном звене (типовое)»  & И6.02  & ГУБиЗИ, ДВА\tabularnewline
\hline 
«\sysabbr{}. Инструкция по эксплуатации комплекса технических средств»  & ИЭ  & ДИС, ЦИТ, ГУБиЗИ \tabularnewline
\hline 
\multicolumn{3}{|l|}{\emph{Приёмо-сдаточная документация \sysabbr{}}}\tabularnewline
\hline 
«\sysabbr{}. Первая очередь. Программа опытной эксплуатации»  & ПЖ  & ДИС, ДВА, ЦИТ, ГУБиЗИ \tabularnewline
\hline 
«\sysabbr{}. Первая очередь. Программа и методика испытаний»  & ПМ  & ДИС, ДВА, ЦИТ, ГУБиЗИ \tabularnewline
\hline 
\end{longtable}

\bigpointРазработка и актуализация документации должна осуществляться
на основе и в соответствии с шаблонами проектной и эксплуатационной
документации системы эксплуатации Банка России.

\bigpointВся разрабатываемая документация должна быть выполнена на
русском языке.

\bigpointТекстовые документы должны быть выполнены в формате PDF\nomenclature{PDF}{Portable Document Format --- стандартный формат для представления электронных документов ISO/IEC 32000-1:2008}.

\newpage{}


\section{Требования к обеспечению конфиденциальности }

Сведения о проведении данной работы, а также другие сведения, ставшие
известными Исполнителю и/или Соисполнителю при проведении работ и
не относящиеся к общеизвестным, являются конфиденциальными и не подлежат
распространению в средствах массовой информации и/или передаче третьим
лицам без письменного разрешения Стороны, являющейся источником и/или
владельцем данных сведений.

\newpage{}

\stepcounter{section}
\renewcommand{\refname}{\thesection\quad{}Источники разработки}

\bibliographystyle{gost780u}
\bibliography{cbr,std}


\newpage{}

\printnomenclature{}
\end{document}
